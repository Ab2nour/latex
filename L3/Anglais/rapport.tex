\documentclass[12pt]{article}
\usepackage[T1]{fontenc}
\usepackage[english]{babel}
\usepackage{microtype}


\usepackage{hyperref}
\hypersetup{
    colorlinks=true,
    linkcolor=blue,
    filecolor=magenta,      
    urlcolor=cyan,
    pdftitle={English Project Report},
    pdfpagemode=FullScreen,
    }

\urlstyle{same}

\usepackage{bookmark}
\hypersetup{hidelinks} %enlève les cadres rouges autour des hyperliens


\begin{document}
    
    \title{Project Report :\\\textbf{New Chess 2\textsuperscript{TM} : Board Game Boogaloo}}
    \author{Abdenour MADANI \and Pierre PIGNÉ}
    \date{January 19\textsuperscript{th} 2022}
    \maketitle
    \newpage
    
    \tableofcontents
    \newpage
    
    \section{Roles Description}
        \subsection{Abdenour}
            \textbf{Product designer and marketing ops.}
            \\He takes care of the overall product user experience and is responsible for the players' satisfaction.
            \\\\So for example, if a player has trouble understanding the game's rules, he has to modify them to make them more explicit. 
            \\\\He is always testing new ideas, prototyping potential better user experiences, measuring players satisfaction through their feedbacks, analyzing data about players during live experiments with volunteers (how well players perform against regular chess, is it more beginner-friendly, ...).
            \\\\He is also part-time involved into marketing operations, trying to make the game go viral via YouTube sponsorships and Twitch streamings.
            
        \subsection{Pierre}
            \textbf{Game designer and programmer.}
            \\He imagined the game rules, has previous experiences with numerous games and video games, so he can come up with lots of new ideas, re-inventing the chess board game.
            \\\\He develops the mobile version of the game, and is in charge of everything related to programming.
            \\\\Every time a bug occurs, he is here to fix it, and he also adds new features like an online scoreboard and the ability to save and share replays among players. 
            \\\\Moreover, he conceives the cards, match their colors through the design he thinks, in order not to impair chess original style.
    
    \section{Project Description}
        \subsection{Introduction}
            What we want to achieve, through our innovative project, is on one hand making chess cool again for young people by updating its old rules so it will pique their curiosity and they may get interested in original chess later, and on the other hand making chess more accessible and fun for normal, non-gifted people who just want to play to unwind.
            \\\\We want people to understand that even an ancient game like chess whose rules never changed is not immutable, and can be modernized, to better suit the needs of today's entertainment industry, but in the meantime, this modern game can go hand in hand with real chess, serving as an introduction for it, as today, either you do not play chess or either or you get demolished by an expert who knows all the classic chess opening lines.
            \\\\We want our project to go viral and launch a new wave of board games, less frustrating, that sparks original situations frequently, without having to be good at the game, as it is mostly handled by luck.
            \\\\On a more serious note, this could benefit people with disabilities, as some might have less ability to concentrate, or even not have the cognitive capacities to play through an entire chess game (which are always intense).
            \\\\So, by basing the games more on luck, one could enjoy those types of game without needing to have strong intellectual capacity.
            \\\\A greater purpose to our project could be developing games for accessibility, for those who cannot enjoy regular games as others do.

        \subsection{Game Description}
            \paragraph{}
                \textbf{New Chess 2\textsuperscript{TM} : Board Game Boogaloo} is a new way of playing Chess by mixing it with other traditional board games like Cards, UNO, Dices and Poker.
            \paragraph{}
                The goal of \textbf{New Chess 2\textsuperscript{TM} : Board Game Boogaloo} is to win a regular game of chess by capturing your opponent's master piece, but with a twist. You can alter the course of the game by playing special cards, taking risky but possibly rewarding bets, and much more.
        \subsection{Game Rules}
            \paragraph{Setup :\\}
                First, setup a traditional $8\times8$ chessboard, as shown in the picture below. Then shuffle all the cards and place them next to the chessboard, accessible to both players, and don't forget to leave an empty space next to the draw pile for the discard pile. Each player draws 5 cards and the game can begin.
            \paragraph{}
                \centerline{TODO : INSERT INITIAL SETUP DIAGRAM}
            \paragraph{Choosing the 1\textsuperscript{st} player :\\}
                Play a best-of-3 game of rock-paper-scissors. The winner decides which player gets to start the game.
            \paragraph{Beginning of a turn :\\}
                At the start of each turn, the player can chose to gamble by choosing "Odd" or "Even", then rolling 5 dices. If they land 3 or more numbers of the chosen type, they can move 2 pieces during their turn instead of 1 (or move the same piece twice). Otherwise, they don't get to play this turn and move on to the next player.
            \paragraph{Possible actions :\\}
                TODO
        \subsection{Cards Description}
            TODO
    
    \section{Pertinent Information}
        Chess is hard$\cite{wegochess, coldwire}$. Not everyone can say they know how to play chess with great skills.
        \\It is also well-known that chess, to some extent, is unfair, as white are deemed to have a slight advantage over black$\cite{fma-wiki}$  (as they start to move first).
        \\That is why we decided to choose the first player to move on random, based on a happy rock-papers-scissors.
    
    \section{Marketing Information}
        Our game is more \textbf{accessible} than regular chess since it does not need as much skill.
        \\\\Furthermore, the game is more \textbf{enjoyable} than regular chess because less brain-demanding, and more lenient : errors are punished less severely than in regular chess, in which one oversight can lead to checkmate.
        \\\\Here there is \textbf{more luck involved}, so it is perfect to play with a friend as a board game during a saturday night, around some beers (Coca-Cola® for children), without the game being at a high disequilibrium, and the more experienced chess player dominating the whole game session.
        \\\\\textbf{Everyone will win at least once} (likely even multiple times during the game session), because the game is rigged to make everyone have fun, and no one be frustrated because they always loose. The victory ratio should near 50/50 because of luck.

    
    \section{SWOT Analysis}
    \textbf{Strengths}
    \\Our game is more accessible than regular chess, based on luck and thus funnier to play.
    \\\\\textbf{Weaknesses}
    \\Our game has even more rules than chess. Even though learning the rules is the easy part$\cite{chessjournal}$, it is still daunting for beginners.
    \\\\\textbf{Opportunities}
    \\We could take advantage of existing chess players on YouTube or Twitch, who are very active and famous, so they may attract a lot of viewers into testing our game. They could even review our game, and make it go viral.
    \\\\\textbf{Threats} 
    \\The major threat is video games. Even though the mobile version of our game could be considered a video game, a game with that much rules could have its growth impeded by video games which could be a lot more funnier, and could have their rules updated more regularly than our chess variation, because we have to support the mobile application and the board game (which would have to be coherent).

    \section{Keywords}
        keywords : chess, board game, game of chance, card game, dice game, UNO, poker, accessibility
    
    \section{Webography}
        \begin{thebibliography}{9}
            \bibitem{fma-wiki}
                Wikipedia, \textit{\href{https://en.wikipedia.org/wiki/First-move_advantage_in_chess}{\underline{First-move advantage in chess}} }
                
            \bibitem{chessjournal}
                Chess Journal, \textit{\href{https://www.chessjournal.com/why-is-chess-so-hard/}{\underline{Why Is Chess So Hard? Is It REALLY That Difficult?}} }
            
            \bibitem{wegochess}
                Michael Stephen Vargas, \href{
        https://wegochess.com/why-is-chess-so-hard-to-learn/}{\underline{\textit{Why is chess so hard to learn}}},
                \\WeGoChess (2017)
            
            \bibitem{coldwire}
                Coldwire,
                \href{https://www.thecoldwire.com/why-is-chess-so-hard/}{\underline{\textit{Why is chess so hard?}}}
        \end{thebibliography}
\end{document}
