\documentclass{article}      % Specifies the document class

\usepackage{amsmath}
\usepackage{amssymb}
\usepackage[noend]{algpseudocode}
\usepackage{algorithm}

% ---------- add foreach ----------
\algnewcommand\algorithmicforeach{\textbf{for each}}
\algdef{S}[FOR]{ForEach}[1]{\algorithmicforeach\ #1\ \algorithmicdo}


% ---------- hack to remove indent ----------
% https://tex.stackexchange.com/questions/354564/how-to-remove-leading-indentation-from-algorithm
\usepackage{xpatch}
\makeatletter
\xpatchcmd{\algorithmic}
  {\ALG@tlm\z@}{\leftmargin\z@\ALG@tlm\z@}
  {}{}
\makeatother


\begin{document}

\section{Arbre couvrant de poids minimum (ACPM)}

\subsection{Définition du problème}
Soit $G = (V, E, w)$ un graphe.
$w$ \textit{(pour weight)} est la fonction de pondération \textit{(le poids)} des arêtes.

$$w : E \mapsto \mathbb{R}^+$$
%
\\
Formellement, trouver un ACPM revient à minimiser 
$$w(H) = \sum_{e \in E(H)} w(e)$$ 
avec $H \subseteq G$.
\\ \textit{($H$ est un arbre couvrant : un graphe partiel de $G$, acyclique maximal)}

\subsection{Méta-algorithme}

- Règle Rouge : Sélectionner un cycle $C$ sans aucune arête rouge.
Colorier en rouge l’arête $e$ sans couleur de $C$ qui a le poids maximum.
\\- Règle Bleue : Sélectionner une coupe $\partial(x)$ qui ne comporte aucune arête bleue.
Colorier en bleu l’arête sans couleur de poids minimum de $\partial(x)$. 
%
$\partial(x)$ L’ensemble des arêtes qui ont une extrémité dans $x$ et l’autre dans $\overline{x}$. 


\subsection{Algorithmes}
% ---------- Kruskal ---------- 
\begin{algorithm} \caption{Kruskal}
\begin{algorithmic}

\State \textbf{Entrée :} $G = (V, E, w)$, un graphe pondéré
\State \textbf{Sortie :} $T = (V, F)$ un Arbre Couvrant de Poids Minimum de $G$

\\ \State \textbf{Complexité :} $O(m\log_2(n) + n + m \cdot \alpha(n,m))$

\\ \State $L =$ la liste des arêtes triées par poids croissant

\\\ForEach {$e \in L$ (pris dans l'ordre)}
    \If {$e$ a ses deux extrémités dans deux Composantes Connexes différentes}
        \State $F := F \cup \{e\}$
    \EndIf
\EndFor

\\ \State \textbf{return} $F$
\end{algorithmic}
\end{algorithm}


% ---------- Jarnik-Prim ---------- 
\begin{algorithm} \caption{Jarnik-Prim}
\begin{algorithmic}

\State \textbf{Entrée :} $G = (V, E, w)$, un graphe pondéré
\State \textbf{Sortie :} $T = (V, F)$ un Arbre Couvrant de Poids Minimum de $G$

\\ \State \textbf{Complexité :} $O(n\log_2(n) + m)$ \textit{avec tas de Fibonacci} 

\\ \State Soit $s$ un sommet de $G$ quelconque.

\\ \State $F := \emptyset$
\\ \State $C := \{s\}$

\\\While {$|C| \neq n-1$}
    \State On applique la règle bleue sur $\partial(C)$
    
    \\ \State Soit $e$ l'arête choisie à l'instruction précédente
    \State $F = F \cup \{e\}$ (\textit{avec $e = \{x, y\}$})
    \State $C = C \cup \{x, y\}$
\EndWhile

\\ \State \textbf{return} $F$
\end{algorithmic}
\end{algorithm}


% ---------- Dijkstra ---------- 
\begin{algorithm} \caption{Dijkstra}
\begin{algorithmic}

\State \textbf{Entrée :} $G = (V, E, w)$, un graphe pondéré et $s$ un sommet de départ
\State \textbf{Sortie :} $T = (V, F)$ un Arbre de Plus Court Chemin

\\ \State \textbf{Complexité :} %todo

\\ \State \textit{Partie 1 : Initialisation}
\ForEach {$v \in V(G)$}
    \State $d[v] := +\infty$
    \State $p[v] := \emptyset$
\EndFor

\\ \State $d[s] := 0$
\State In $:= \emptyset$
\State Out $:= V(G)$

\\\\ \State \textit{Partie 2 : Calcul des plus courts chemins}
\While {$|$In$| < n$}
    \State Soit $x$ le sommet de Out avec la plus petite valeur $d[x]$
    \State In $:=$ In $\cup \{x\}$
    \State Out $:=$ Out$\backslash \{x\}$
    
    \\ \ForEach {$z \in N(x)$}
        \If {$(d[z] > d[x] + w(x, z)$}
            \State $d[v] := d[x] + w(x, z)$
            \State $p[z] := x$
        \EndIf
    \EndFor
\EndWhile

\\ \State \textbf{return} $p$
\end{algorithmic}
\end{algorithm}


% ---------- Bellman-Ford ---------- 
\begin{algorithm} \caption{Bellman \textit{(Bellman-Ford)}}
\begin{algorithmic}

\State \textbf{Entrée :} $G = (V, E)$, un graphe orienté valué et un sommet $s$
\State \textbf{Sortie :} $T$ une arborescence de plus court chemin ou Impossible si pas possible

\\ \State \textbf{Complexité :} $O(n\times m)$

\\ \State \textit{Partie 1 : Initialisation}
\ForEach {$v \in V(G)$}
    \State $d[v] := +\infty$
    \State $p[v] := \emptyset$
\EndFor


\\ \State \textit{Partie 2 : Calcul des plus courts chemins}
\For {$i=1$ to $|V(G)|-1$}
    \ForEach {arc $(u, v)$}
        \If {$(d[v] > d[u] + w(u ,v))$}
            \State $d[v] := d[u] + w(u, v)$
            \State $p[v] := u$
        \EndIf
    \EndFor
\EndFor


\\ \State \textit{Partie 3 : Détection de cycle de poids négatif}
\ForEach {arc $(u, v)$}
    \If {$(d[u] > d[u] + w(u ,v))$}
    \State \textbf{return} Impossible \textit{(cycle de poids négatif)}
    \EndIf
\EndFor

\\ \State \textbf{return} $p$
\end{algorithmic}
\end{algorithm}


% ---------- Edmonds-Karp ---------- 
\begin{algorithm} \caption{Edmonds-Karp}
\begin{algorithmic}

\\ \State Au lieu de prendre n’importe quel chemin améliorant de $s$ à $t$, on prend le plus court en nombre d’arcs et on fait un BFS sur le graphe résiduel $G_f$.

\\ \State \textbf{Entrée :} $G = (V, E, c)$, un graphe avec une capacité pour chaque arc
\State \textbf{Sortie :} $f$, le flot maximum

\\ \State \textbf{Complexité :} %todo

\\ \While {$\exists P$ un chemin $f-$améliorant dans $G$}
    \State On augmente le flot le long de $P$
\EndWhile

\\ \State \textbf{return} $f$
\end{algorithmic}
\end{algorithm}

\end{document}
