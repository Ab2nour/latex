\documentclass{article}


% -------------------- Packages --------------------
\usepackage[utf8]{inputenc}
\usepackage[T1]{fontenc}%
\usepackage{lmodern}%
\usepackage{textcomp}%
\usepackage{lastpage}%


\usepackage{amsmath}
\usepackage{amssymb}
\usepackage{graphicx}
\usepackage{float}


\usepackage{soul} % highlighting
\usepackage{inconsolata} % monospace font
\usepackage{minted}
\usepackage{fontawesome5}


\usepackage{xcolor}
\usepackage[framemethod=tikz]{mdframed}
\usepackage{tikzpagenodes}
\usetikzlibrary{calc}


\usepackage{hyperref}
\hypersetup{
    colorlinks=true,
    linkcolor=blue,
    filecolor=magenta,      
    urlcolor=cyan,
    pdftitle={Clovis LaTeX Package},
    pdfpagemode=FullScreen,
    }

\urlstyle{same}

\usepackage{bookmark}
\hypersetup{hidelinks}


% -------------------- Colors --------------------
\definecolor{seance-color}{HTML}{2f80ed}
\definecolor{seance-bg-color}{HTML}{dfebfc}

\definecolor{example-color}{HTML}{6b7280}
\definecolor{example-bg-color}{HTML}{eceef2}

\definecolor{byheart-color}{HTML}{ffb0da}
\definecolor{byheart-bg-color}{HTML}{fff3f9}

\definecolor{danger-color}{HTML}{e6505f}
\definecolor{danger-bg-color}{HTML}{fbe4e7}

\definecolor{definition-color}{HTML}{47c2bb}
\definecolor{definition-bg-color}{HTML}{e3f5f4}

\definecolor{code-bg-color}{HTML}{f3f4f6} % todo: temp color
\definecolor{code-border-color}{HTML}{dadce0} % todo: temp color

\definecolor{hl-yellow-color}{HTML}{fef3c7}


% -------------------- Macros --------------------
% highlight function
\newcommand{\highlight}[2]{%
    \begingroup
    \sethlcolor{#1}%
    \hl{#2}%
    \endgroup
}
\newcommand{\hlYellow}[1]{%
    \highlight{hl-yellow-color}{#1}%
}
% inlineCode (without border)
\newcommand{\inlineCodeWithoutBorder}[1]{%
    {\small\tt \highlight{code-bg-color}{#1}}%
}
% inlineCode (with border)
\usepackage[most]{tcolorbox}
\tcbset{
    on line,
    boxsep=2px,
    left=0pt,
    right=0pt,
    top=0pt,
    bottom=0pt,
    boxrule=0.5px,
    colframe=code-border-color,
    colback=code-bg-color,
    highlight math style={enhanced},
    breakable
}
\newcommand{\inlineCode}[1]{%
    \tcbox{{\small\tt #1}}%
}
% -------------------- colorful-blocks --------------------
\mdfdefinestyle{definition-style}{%
innertopmargin=10px,
innerbottommargin=10px,
linecolor=definition-color,
backgroundcolor=definition-bg-color,
linewidth=2px,
topline=false,
bottomline=false,
rightline=false,
}

\newcommand\clovisColorfulBlock[4]{
    % #1 = danger (name)
    % #2 = Danger (color)
    % #3 = danger-bg-color (background color)
    \mdfdefinestyle{#1-style}{%
        innertopmargin=10px,
        innerbottommargin=10px,
        innerrightmargin=12px,
        linecolor=#1-color,
        backgroundcolor=#1-bg-color,
        linewidth=2px,
        topline=false,
        bottomline=false,
        rightline=false,
    }
    \newmdenv[style=#1-style]{#1}
    \expandafter\newcommand\csname clovis#2\endcsname[1]{
        \begin{#1}
        {\scriptsize \textcolor{#1-color}{\faIcon{#3} \textbf{#4}}}
        \vspace{6px}
        \\ ##1
        \end{#1}
    }
}
\clovisColorfulBlock{seance}{Seance}{running}{SÉANCE}
\clovisColorfulBlock{definition}{Definition}{graduation-cap}{DÉFINITION}



\newcommand{\titre}[1]{%
    \MakeUppercase{ \texttt{#1}}%
}

\newcommand{\serie}{%
    \titre{Série   : }
}

\newcommand{\vitesse}{%
    \titre{Vitesse : }%
}

\newcommand{\pause}{%
    \titre{Pause   : }%
}

\newcommand{\recup}{%
    \titre{Récup.  : }%
}
% -------------------- Study Sheet --------------------
\title{Carnet d'Entraînement\\
UE Libre Running}%
\author{MADANI Abdenour}%
\date{\today}%
\normalsize%
%
\setcounter{tocdepth}{4}
\setcounter{secnumdepth}{4}
%
\begin{document}%
\normalsize%
\maketitle%
\tableofcontents%
\newpage%

\maketitle

\section{Introduction}
    Ceci est mon carnet d'entraînement pour l'UE Libre Running.\\
    Je vais commencer par décrire un échauffement typique.\\
    Je parlerai ensuite dans le détail de chacune des séances, dont je décrirai le programme, puis les résultats que j'ai eus lors de cette séance, ainsi que mon ressenti.\\
    Je ferai ensuite un bilan de tout ce que j'ai appris au long de cette UE.\\\\
    Je définis ici un terme central pour cette UE, et la course en général, et qui reviendra très souvent par la suite, la \textbf{VMA}.
    \clovisDefinition{
                {\underline{\textbf{VMA}}\vspace{3px}}\\
                Vitesse Maximum Aérobie.\\
                C'est la plus petite vitesse à partir de laquelle la consommation d'oxygène est maximale.
    
    }

\section{Échauffement}
    Je fais 12-20 minutes à allure footing.\\
    Je commence toujours très lentement, afin de m'échauffer et de monter en allure progressivement.\\\\
    Je fais généralement 2 tours à 2'00'' (9 km/h), avant de monter petit à petit jusqu'à 1'35'' (11 km/h) au tour, rythme que j'arrive à tenir 20 minutes, comme je le décris dans la \underline{\hyperref[sec:seance7]{séance 7 {\scriptsize \faIcon{external-link-alt}}}}.\\\\
    Je continue par des gammes : montée de genoux, talons-fesses, vite relâché vite.\\
    Optionnellement, je fais parfois des rebonds (pour bien apprendre à rebondir sur la pointe des pieds), ou aussi des 50m à l'allure désirée pour le début de la séance, afin de bien retrouver mes repères.\\
    expliquer que c'est pour avoir le bon mouvement quand on court
    (rebondir sur la pointe des pieds et pas le talon)
    5 lignes droites
    
    La récupération en fin de séance consiste toujours en une dizaine de minutes à allure footing.

\section{Séances}
    \subsection{1$^{\text{ère}}$ séance (20.01)}
        \subsubsection*{Description de la séance}
            \clovisSeance{
                \underline{\textbf{Objectif :}}
                \begin{itemize}
                    \item Chercher sa VMA
                \end{itemize}
                %
                \underline{\textbf{Programme de la séance :}}
                \begin{itemize}
                    \item \serie 6 $\times$ 36'' 
                    \\\vitesse VMA$^+$
                    \\\pause 36'' (marche)
                    
                    \item \serie 5 $\times$ 1'12'' 
                    \\\vitesse VMA
                    \\\pause 1' (marche + trot)
                    
                    \item \serie 2 $\times$ 4'00''
                    \\\vitesse Seuil$^+$
                    \\\pause 3' (marche)
                \end{itemize}
            }
        
        \subsubsection*{Résultats de la séance}
    
    \subsection{27.01}
        absent (malade)
    
    \subsection{3$^{\text{ème}}$ séance (03.02)}
        \subsubsection*{Description de la séance}
            \clovisSeance{
                \underline{\textbf{Objectif :}}
                \begin{itemize}
                    \item Confirmer sa VMA
                \end{itemize}
                %
                \underline{\textbf{Programme de la séance :}}
                \begin{itemize}
                    \item \serie 5 $\times$ 1'12'' 
                    \\\vitesse VMA
                    \\\pause 1' (marche)
                    
                    \item \recup 3'00'' 
                    
                    \item \serie 5 $\times$ 1'12'' 
                    \\\vitesse VMA
                    \\\pause 1' (marche)
                \end{itemize}
            }
        
        \subsubsection*{Résultats de la séance}
    
    
    \subsection{4$^{\text{ème}}$ séance, test VMA n°1 (10.02)}
        \subsubsection*{Description de la séance}
            \clovisSeance{
                \underline{\textbf{Objectif :}}
                \begin{itemize}
                    \item Test : tenir sa VMA pendant 4 minutes
                \end{itemize}
                %
                \underline{\textbf{Programme de la séance :}}
                \begin{itemize}
                    \item \serie 1 $\times$ 4'00'' 
                    \\\vitesse VMA
                    
                    \item \recup 15'00'' 
                    
                    \item \serie 5 $\times$ 1'12'' 
                    \\\vitesse VMA
                    \\\pause 1' (marche)
                \end{itemize}
            }
        test 4 minutes 1 km
    
    
    \subsection{5$^{\text{ème}}$ séance (17.02)}
        %TODO
        \subsubsection*{Description de la séance}
            \clovisSeance{
                \underline{\textbf{Objectif :}}
                \begin{itemize}
                    \item TODO
                \end{itemize}
                %
                \underline{\textbf{Programme de la séance :}}
                \begin{itemize}
                    \item \serie 5 $\times$ 1'12'' 
                    \\\vitesse VMA
                    \\\pause 1' (marche)
                    
                    \item \recup 3'00'' 
                \end{itemize}
            }
        
        \subsubsection*{Résultats de la séance}
            300/250 m à VMA+ (16 KM)
            1min02
            Dernier tour 56 s
    
    
    \subsection{6$^{\text{ème}}$ séance (03.03)}
        %TODO
        \subsubsection*{Description de la séance}
            \clovisSeance{
                \underline{\textbf{Objectif :}}
                \begin{itemize}
                    \item TODO
                \end{itemize}
                %
                \underline{\textbf{Programme de la séance :}}
                \begin{itemize}
                    \item \serie 5 $\times$ 1'12'' 
                    \\\vitesse VMA
                    \\\pause 1' (marche)
                    
                    \item \recup 3'00'' 
                \end{itemize}
            }
        
        \subsubsection*{Résultats de la séance}
        Running 3*5 * 200 m
        47 s
        Max 45 
        Min 49 (hum)
        
        À VMA plus (16 km h)
        
        
    \subsection{7$^{\text{ème}}$ séance (10.03)}
    \label{sec:seance7}
        %TODO
        \subsubsection*{Description de la séance}
            \clovisSeance{
                \underline{\textbf{Objectif :}}
                \begin{itemize}
                    \item TODO
                \end{itemize}
                %
                \underline{\textbf{Programme de la séance :}}
                \begin{itemize}
                    \item \serie 5 $\times$ 1'12'' 
                    \\\vitesse VMA
                    \\\pause 1' (marche)
                    
                    \item \recup 3'00'' 
                \end{itemize}
            }
        
        \subsubsection*{Résultats de la séance}
        13 tours et demi (objectif de 13 au début)
        à 11 km/h (recalculer, vitesse seuil ) pendant 20 minutes
    
    
    \subsection{8$^{\text{ème}}$ séance (17.03)}
        %TODO
        \subsubsection*{Description de la séance}
            \clovisSeance{
                \underline{\textbf{Objectif :}}
                \begin{itemize}
                    \item TODO
                \end{itemize}
                %
                \underline{\textbf{Programme de la séance :}}
                \begin{itemize}
                    \item \serie 5 $\times$ 1'12'' 
                    \\\vitesse VMA
                    \\\pause 1' (marche)
                    
                    \item \recup 3'00'' 
                \end{itemize}
            }
        
        \subsubsection*{Résultats de la séance}
        Ici objectif tenter plus que VMA (0.5 ou 1 km/h)
        300 m vitesse 15/16
        1.09 min 1.07 max 1.11
        250 m vitesse idem
        55 min 57 en moyenne max 59
        
        
    \subsection{9$^{\text{ème}}$ séance (24.03)}
        %TODO
        \subsubsection*{Description de la séance}
            \clovisSeance{
                \underline{\textbf{Objectif :}}
                \begin{itemize}
                    \item TODO
                \end{itemize}
                %
                \underline{\textbf{Programme de la séance :}}
                \begin{itemize}
                    \item \serie 5 $\times$ 1'12'' 
                    \\\vitesse VMA
                    \\\pause 1' (marche)
                    
                    \item \recup 3'00'' 
                \end{itemize}
            }
        
        \subsubsection*{Résultats de la séance}
        5 * 300 VMA
        1.08 sauf deux 1.10 et 1.12
        
        Puis 3 * 4 minutes seuil+ (13)
        Objectif 1.19 voire 1.17 au tour
        2 fois objectif accompli, une fois boffff
        
        
    \subsection{10$^{\text{ème}}$ séance (04.04)}
        %TODO
        \subsubsection*{Description de la séance}
            \clovisSeance{
                \underline{\textbf{Objectif :}}
                \begin{itemize}
                    \item TODO
                \end{itemize}
                %
                \underline{\textbf{Programme de la séance :}}
                \begin{itemize}
                    \item \serie 5 $\times$ 1'12'' 
                    \\\vitesse VMA
                    \\\pause 1' (marche)
                    
                    \item \recup 3'00'' 
                \end{itemize}
            }
        
        \subsubsection*{Résultats de la séance}
        Idem que 24.03
        
        
    \subsection{11$^{\text{ème}}$ séance, test VMA n°2 (07.04)}
        %TODO
        \subsubsection*{Description de la séance}
            \clovisSeance{
                \underline{\textbf{Objectif :}}
                \begin{itemize}
                    \item TODO
                \end{itemize}
                %
                \underline{\textbf{Programme de la séance :}}
                \begin{itemize}
                    \item \serie 5 $\times$ 1'12'' 
                    \\\vitesse VMA
                    \\\pause 1' (marche)
                    
                    \item \recup 3'00'' 
                \end{itemize}
            }
        
        \subsubsection*{Résultats de la séance}
        test 4 minutes
        observations : objectif accompli, mais manque de régularité
        je suis parti trop vite
        
        
    \subsection{12$^{\text{ème}}$ séance, test Seuil$^{\text{+}}$ (14.04)}
        %TODO
        \subsubsection*{Description de la séance}
            \clovisSeance{
                \underline{\textbf{Objectif :}}
                \begin{itemize}
                    \item TODO
                \end{itemize}
                %
                \underline{\textbf{Programme de la séance :}}
                \begin{itemize}
                    \item \serie 5 $\times$ 1'12'' 
                    \\\vitesse VMA
                    \\\pause 1' (marche)
                    
                    \item \recup 3'00'' 
                \end{itemize}
            }
        
        \subsubsection*{Résultats de la séance}
        test 12 minutes
        
\section{Ressenti personnel et progression}
    Donner mes impressions, mon ressenti
    
    
    dire où j'en étais *avant* et ce que j'ai tiré de l'UE libre :
    - j'ai beaucoup appris sur la façon de courir
    les différentes allures
    le bon mouvement (j'ai un meilleur mouvement désormais)
    j'ai définitivement compris les repères sur la piste
    
    avant je courais sans comprendre les séances de Running Loisir, maintenant je saisis mieux l'objectif de chaque séance
    Je saurais désormais travailler dans le cadre d'un objectif personnel, j'ai construit mes propres repères (différentes allures, avec le volume, l'intensité et la récupération qui vont avec)
    
    
    
    dire les sensations (manque de souffle ? jambes courbaturées ?)
    et aussi dire que je me sens meilleur au sprint / sur du court
    que sur du long, donc que plus tard
    je devrais travailler mes faiblesses sur le long/endurance
    et renforcer mes facilités en sprint tout ça
\end{document}
