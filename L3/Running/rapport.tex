\documentclass{article}


% -------------------- Packages --------------------
\usepackage[utf8]{inputenc}
\usepackage[T1]{fontenc}%
\usepackage{lmodern}%
\usepackage{textcomp}%
\usepackage{lastpage}%


\usepackage{amsmath}
\usepackage{amssymb}
\usepackage{graphicx}
\usepackage{float}


\usepackage{soul} % highlighting
\usepackage{inconsolata} % monospace font
\usepackage{minted}
\usepackage{fontawesome5}


\usepackage{xcolor}
\usepackage[framemethod=tikz]{mdframed}
\usepackage{tikzpagenodes}
\usetikzlibrary{calc}


\usepackage{hyperref}
\hypersetup{
    colorlinks=true,
    linkcolor=blue,
    filecolor=magenta,      
    urlcolor=cyan,
    pdftitle={Clovis LaTeX Package},
    pdfpagemode=FullScreen,
    }

\urlstyle{same}

\usepackage{bookmark}
\hypersetup{hidelinks}

\usepackage{tabularx}

% -------------------- Colors --------------------
\definecolor{seance-color}{HTML}{2f80ed}
\definecolor{seance-bg-color}{HTML}{dfebfc}

\definecolor{example-color}{HTML}{6b7280}
\definecolor{example-bg-color}{HTML}{eceef2}

\definecolor{byheart-color}{HTML}{ffb0da}
\definecolor{byheart-bg-color}{HTML}{fff3f9}

\definecolor{danger-color}{HTML}{e6505f}
\definecolor{danger-bg-color}{HTML}{fbe4e7}

\definecolor{definition-color}{HTML}{47c2bb}
\definecolor{definition-bg-color}{HTML}{e3f5f4}

\definecolor{code-bg-color}{HTML}{f3f4f6} % todo: temp color
\definecolor{code-border-color}{HTML}{dadce0} % todo: temp color

\definecolor{hl-yellow-color}{HTML}{fef3c7}


% -------------------- Macros --------------------
% highlight function
\newcommand{\highlight}[2]{%
    \begingroup
    \sethlcolor{#1}%
    \hl{#2}%
    \endgroup
}
\newcommand{\hlYellow}[1]{%
    \highlight{hl-yellow-color}{#1}%
}
% inlineCode (without border)
\newcommand{\inlineCodeWithoutBorder}[1]{%
    {\small\tt \highlight{code-bg-color}{#1}}%
}
% inlineCode (with border)
\usepackage[most]{tcolorbox}
\tcbset{
    on line,
    boxsep=2px,
    left=0pt,
    right=0pt,
    top=0pt,
    bottom=0pt,
    boxrule=0.5px,
    colframe=code-border-color,
    colback=code-bg-color,
    highlight math style={enhanced},
    breakable
}
\newcommand{\inlineCode}[1]{%
    \tcbox{{\small\tt #1}}%
}
% -------------------- colorful-blocks --------------------
\mdfdefinestyle{definition-style}{%
innertopmargin=10px,
innerbottommargin=10px,
linecolor=definition-color,
backgroundcolor=definition-bg-color,
linewidth=2px,
topline=false,
bottomline=false,
rightline=false,
}

\newcommand\clovisColorfulBlock[4]{
    % #1 = danger (name)
    % #2 = Danger (color)
    % #3 = danger-bg-color (background color)
    \mdfdefinestyle{#1-style}{%
        innertopmargin=10px,
        innerbottommargin=10px,
        innerrightmargin=12px,
        linecolor=#1-color,
        backgroundcolor=#1-bg-color,
        linewidth=2px,
        topline=false,
        bottomline=false,
        rightline=false,
    }
    \newmdenv[style=#1-style]{#1}
    \expandafter\newcommand\csname clovis#2\endcsname[1]{
        \begin{#1}
        {\scriptsize \textcolor{#1-color}{\faIcon{#3} \textbf{#4}}}
        \vspace{6px}
        \\ ##1
        \end{#1}
    }
}
\clovisColorfulBlock{seance}{Seance}{running}{SÉANCE}
\clovisColorfulBlock{definition}{Definition}{graduation-cap}{DÉFINITION}



\newcommand{\titre}[1]{%
    \MakeUppercase{ \texttt{#1}}%
}

\newcommand{\serie}{%
    \titre{Série   : }
}

\newcommand{\vitesse}{%
    \titre{Vitesse : }%
}

\newcommand{\pause}{%
    \titre{Pause   : }%
}

\newcommand{\recup}{%
    \titre{Récup.  : }%
}
% -------------------- Study Sheet --------------------
\title{Carnet d'Entraînement\\
UE Libre Running}%
\author{MADANI Abdenour}%
\date{\today}%
\normalsize%
%
\setcounter{tocdepth}{4}
\setcounter{secnumdepth}{4}
%
\begin{document}%
\normalsize%
\maketitle%
\tableofcontents%
\newpage%

\maketitle

\section{Introduction}
    Ceci est mon carnet d'entraînement pour l'UE Libre Running.\\
    Je vais commencer par décrire un échauffement typique.\\
    Je parlerai ensuite dans le détail de chacune des séances, dont je décrirai le programme, puis les résultats que j'ai eus lors de cette séance, ainsi que mon ressenti.\\
    Je ferai ensuite un bilan de tout ce que j'ai appris au long de cette UE.\\\\
    Je définis ici un terme central pour cette UE, et la course en général, et qui reviendra très souvent par la suite, la \textbf{VMA}.
    \clovisDefinition{
                {\underline{\textbf{VMA}}\vspace{3px}}\\
                Vitesse Maximum Aérobie.\\
                C'est la plus petite vitesse à partir de laquelle la consommation d'oxygène est maximale.
    
    }

\section{Échauffement}
    On fait 12-20 minutes à allure footing : je commence toujours très lentement, afin de m'échauffer et de monter en allure progressivement.\\\\
    Je fais généralement 2 tours à 2'00'' (9 km/h), avant de monter petit à petit jusqu'à 1'35'' (11 km/h) au tour, rythme que j'arrive à tenir 20 minutes, comme je le décris dans la \underline{\hyperref[sec:seance7]{séance 7 {\scriptsize \faIcon{external-link-alt}}}}.\\\\
    %
    On continue par des gammes : montée de genoux, talons-fesses, "vite relâché vite" (3 à 5 fois).\\
    La montée de genoux et le talons-fesse servent à gagner en vélocité, mais en pratique, sur nos allures, on n'utilisera que le mouvement du talons-fesse, à partir de VMA$^+$.\\
    Le "vite relâché vite" sert à nous habituer à changer d'allure et à mieux nous rendre compte des variations de vitesse, notamment afin de mieux reprendre de la vitesse quand on commence à perdre du régime en fin de série.\\\\
    %
    Optionnellement, on fait parfois des lignes droites en "rebonds" (pour bien apprendre à rebondir sur la pointe des pieds), ou aussi des 50 m à l'allure désirée pour le début de la séance, afin de bien retrouver nos repères.\\\\
    %
    La récupération en fin de séance consiste toujours en une dizaine de minutes à allure footing.%
%

\section{Séances}
    \subsection{1$^{\text{ère}}$ séance (20.01)}
        \subsubsection*{Description de la séance}
            \clovisSeance{
                \underline{\textbf{Objectifs :}}
                \begin{itemize}
                    \item Chercher sa VMA
                    \item (Re)découvrir différentes allures
                \end{itemize}
                %
                \underline{\textbf{Programme de la séance :}}
                \begin{itemize}
                    \item \serie 6 $\times$ 36'' 
                    \\\vitesse VMA$^+$
                    \\\pause 36'' (marche)
                    
                    \item \serie 5 $\times$ 1'12'' 
                    \\\vitesse VMA
                    \\\pause 1' (marche + trot)
                    
                    \item \serie 2 $\times$ 2.5 tours
                    \\\vitesse Seuil$^+$
                    \\\pause 3' (marche)
                \end{itemize}
            }
        
        \subsubsection*{Résultats de la séance}
            Ayant déjà fait du running en loisir pendant 1 semestre et demi, j'ai une petite idée de mes repères. Toutefois, je n'ai jusqu'à présent pas forcément compris tout ce que je faisais, et je suivais plutôt les autres.\\\\
            Je pense que ma VMA se situe autour de 15 km/h.\\\\
            %
            Je tente avec la VMA$^+$ à 16 km/h sur 36'' (= 0.01 h), objectif 160 m (donc 1 demi-tour + 20 m).\\
            J'ai un très bon ressenti, c'est sur cet exercice que je me sens le plus à l'aise.\\
            J'arrive exactement au même endroit à chaque fois, le temps de récupération est serré mais suffisant. Je m'en sors bien.\\\\
            %
            Pour les séries à VMA (15 km/h) : je suis arrivé au plot 15 (300 m) les trois premières fois, puis entre les plots 14 et 15 les deux dernières fois.\\
            C'est correct au début, mais je commence à lâcher prise passé la troisième fois...\\\\
            %
            Pour le Seuil$^+$, c'est encore pire : je suis théoriquement à 13.5 km/h (1'16'' au tour)... cependant, j'ai fait moins, avec 1'25'' au tour, soit 12 km/h.\\
            Le ressenti était atroce, j'étais au bord de l'échec, pour un résultat insatisfaisant. Certes, c'est la fin de séance, mais je sens clairement que plus la durée de course est élevée, plus je suis en difficulté.
    %
    \subsection{2$^{\text{ème}}$ séance (27.01)}
        Absent (malade).
    
    \subsection{3$^{\text{ème}}$ séance (03.02)}
        \subsubsection*{Description de la séance}
            \clovisSeance{
                \underline{\textbf{Objectif :}}
                \begin{itemize}
                    \item Confirmer sa VMA
                \end{itemize}
                %
                \underline{\textbf{Programme de la séance :}}
                \begin{itemize}
                    \item \serie 5 $\times$ 1'12'' 
                    \\\vitesse VMA
                    \\\pause 1' (marche)
                    
                    \item \recup 3'00'' 
                    
                    \item \serie 5 $\times$ 1'12'' 
                    \\\vitesse VMA
                    \\\pause 1' (marche)
                \end{itemize}
            }
        
        \subsubsection*{Résultats de la séance}
            Il s'agit d'une séance de VMA classique, je suis parti avec un objectif de 15 km/h.\\
            J'ai réussi les 4 premiers (sur 5) pour les 2 séries de 5 répétitions, puis sur la dernière j'arrive entre les deux plots de 14 et 15 km/h.\\
            J'ai finalement fait 2 répétitions de plus (après récupération de 3'00''), à 15 km/h aussi.\\
            Ma VMA de 15 km/h est bien confirmée.
    
    
    \newpage
    
    \subsection{4$^{\text{ème}}$ séance, test VMA n°1 (10.02)}
        \subsubsection*{Description de la séance}
            \clovisSeance{
                \underline{\textbf{Objectif :}}
                \begin{itemize}
                    \item Test : tenir sa VMA pendant 4 minutes
                \end{itemize}
                %
                \underline{\textbf{Programme de la séance :}}
                \begin{itemize}
                    \item \serie 1 $\times$ 4'00'' 
                    \\\vitesse VMA
                    
                    \item \recup 15'00'' 
                    
                    \item \serie 5 $\times$ 200 m
                    \\\vitesse VMA$^+$
                    \\\pause temps de course (marche)
                \end{itemize}
            }
        
        \subsubsection*{Résultats de la séance}
            Mon objectif pour le test était bien entendu 15 km/h.\\
            Voici mon tableau pour le test :\\\\
\begin{tabularx}{1\textwidth} { 
  | >{\raggedright\arraybackslash}X 
  | >{\raggedleft\arraybackslash}X 
  | >{\raggedleft\arraybackslash}X 
  | >{\raggedleft\arraybackslash}X 
  | >{\raggedleft\arraybackslash}X 
  | >{\raggedleft\arraybackslash}X | }
 \hline
 Distance{\scriptsize (m)} & 200 & 400 & 600 & 800 & 1000 \\
 \hline
 Temps & 0'45''  & 1'36'' & 2'22'' & 3'11'' & 4'00''  \\
 \hline
 Laps & 45'' & 51'' & 46'' & 49'' & 49''  \\
\hline
\end{tabularx}
            \\\\\\Je suis très content, car j'ai réussi à faire pile 1 km, ce qui correspond \textbf{exactement} à ma VMA de 15 km/h.\\\\
            Je devrais faire 48'' au tour (200 m).\\
            On observe cependant que je pars légèrement trop vite, puis je fais le deuxième tour légèrement trop lent (-3'' et +3'' respectivement : ça se compense !).\\
            Troisième tour : je reprends un peu de vitesse pour rattraper le tour d'avant.\\\\
            Puis on observe que durant les deux derniers tours je suis à un rythme de croisière, régulier à 49''.\\\\
            Niveau sensations, je me suis donné vraiment à fond, je ne pensais jamais tenir 4 minutes à ma VMA, j'ai été agréablement surpris par ma performance (je pensais au contraire, comme pour les séances, réussir les premiers tours puis progressivement perdre en vitesse).\\
            Après les 4 minutes, j'étais complètement KO, la gorge et les poumons en feu, mais le résultat est satisfaisant et très motivant.\\
            Ayant réussi du premier coup les 15 km/h, c'est signe que pour l'évaluation je pourrai viser au moins 15.5 km/h.
    
    
    \subsection{5$^{\text{ème}}$ séance (17.02)}
    \label{sec:seance5}
        %TODO
        \subsubsection*{Description de la séance}
            \clovisSeance{
                \underline{\textbf{Objectifs :}}
                \begin{itemize}
                    \item Commencer sur du 300 m, puis continuer sur du 250 m à la même vitesse, pour se sentir plus à l'aise/pousser plus loin à VMA$^+$.
                    \item Augmenter sa VMA en tentant une allure plus rapide que sa VMA sur 300 m (la distance de course habituelle pour la VMA)
                \end{itemize}
                %
                \underline{\textbf{Programme de la séance :}}
                \begin{itemize}
                    \item \serie 4 $\times$ 300 m
                    \\\vitesse VMA$^+$
                    \\\pause temps - 10''
                    
                    \item \recup 2'30'' 
                    
                    \item \serie 4 $\times$ 250 m
                    \\\vitesse VMA$^+$
                    \\\pause temps - 10''
                    
                    \item \recup 2'30'' 
                    
                    \item \serie 4 $\times$ 250 m
                    \\\vitesse VMA$^+$
                    \\\pause temps - 10''
                \end{itemize}
            }
        
        \subsubsection*{Résultats de la séance}
            Mon objectif de VMA$^+$ est de 16 km/h, soit 1'07'' pour le 300 m et 0'56'' pour le 250 m.
            300/250 m à VMA+ (16 KM)
            1min02 Dernier tour 300 m
            56 s
    
    
    \subsection{6$^{\text{ème}}$ séance (03.03)}
        %TODO
        \subsubsection*{Description de la séance}
            \clovisSeance{
                \underline{\textbf{Objectif :}}
                \begin{itemize}
                    \item Faire progresser sa VMA en courant uniquement plus vite
                \end{itemize}
                %
                \underline{\textbf{Programme de la séance :}}
                \begin{itemize}
                    \item \serie 5 $\times$ 200 m
                    \\\vitesse VMA$^+$
                    \\\pause temps de course (marche)
                    
                    \item \recup 3'00'' 
                    
                    \textbf{Le tout $\times$3}
                \end{itemize}
            }
        
        \subsubsection*{Résultats de la séance}
        obj 45 = 16km/h
        47 presque = 15 km/h
        Running 3*5 * 200 m
        47 s
        Max 45 
        Min 49 (hum)
        
        À VMA plus (16 km h)
        
        
    \subsection{7$^{\text{ème}}$ séance (10.03)}
    \label{sec:seance7}
        %TODO
        \subsubsection*{Description de la séance}
            \clovisSeance{
                \underline{\textbf{Objectifs :}}
                \begin{itemize}
                    \item Tenir l'allure  footing pendant 20 minutes
                    \item Valider son allure footing par rapport à sa VMA
                    \item Faire suivre le footing par une série à allure VMA
                \end{itemize}
                %
                \underline{\textbf{Programme de la séance :}}
                \begin{itemize}
                    \item \serie 1 $\times$ 20'00''
                    \\\vitesse Footing
                    
                    \item \recup 5'00'' 
                    
                    \item \serie 5 $\times$ 1'12'' 
                    \\\vitesse VMA
                    \\\pause 1' (marche)
                \end{itemize}
            }
        
        \subsubsection*{Résultats de la séance}
        13 tours et demi (objectif de 13 au début)
        à 11 km/h (recalculer, vitesse seuil ) pendant 20 minutes
        le mental compte beaucoup (pas que le physique)
        couru à plusieurs (effet de boost !)
    
    \subsection{8$^{\text{ème}}$ séance (17.03)}
        %TODO
        \subsubsection*{Description de la séance}
            \clovisSeance{
                \underline{\textbf{Objectifs :}}
                \begin{itemize}
                    \item Commencer sur du 300 m, puis continuer sur du 250 m à la même vitesse, pour se sentir plus à l'aise/pousser plus loin à VMA$^+$.
                    \item Augmenter sa VMA en tentant une allure plus rapide que sa VMA sur 300 m (la distance de course habituelle pour la VMA)
                \end{itemize}
                (Identique à la \underline{\hyperref[sec:seance5]{séance 5 {\scriptsize \faIcon{external-link-alt}}}}).\\\\
                %
                \underline{\textbf{Programme de la séance :}}
                \begin{itemize}
                    \item \serie 4 $\times$ 300 m
                    \\\vitesse VMA$^+$
                    \\\pause temps - 10''
                    
                    \item \recup 2'30'' 
                    
                    \item \serie 4 $\times$ 250 m
                    \\\vitesse VMA$^+$
                    \\\pause temps - 10''
                    
                    \item \recup 2'30'' 
                    
                    \item \serie 4 $\times$ 250 m
                    \\\vitesse VMA$^+$
                    \\\pause temps - 10''
                \end{itemize}
            }
        
        \subsubsection*{Résultats de la séance}
        Ici objectif tenter plus que VMA (0.5 ou 1 km/h)
        300 m vitesse 15/16
        1.09 min 1.07 max 1.11
        250 m vitesse idem
        55 min 57 en moyenne max 59
        
        
    \subsection{9$^{\text{ème}}$ séance (24.03)}
    \label{sec:seance9}
        \subsubsection*{Description de la séance}
            \clovisSeance{
                \underline{\textbf{Objectifs :}}
                \begin{itemize}
                    \item Préparation aux 2 tests de l'UE Libre (à allure VMA et Seuil$^+$)
                    \item Passer de l'allure VMA à Seuil$^+$ pour se sentir plus léger
                    \item Se prouver qu'on peut courir (au total) 12 minutes à Seuil$^+$, même \textit{après} plusieurs 300 m à VMA
                \end{itemize}
                %
                \underline{\textbf{Programme de la séance :}}
                \begin{itemize}
                    \item \serie 5 $\times$ 300 m
                    \\\vitesse VMA
                    \\\pause 1' (marche)
                    
                    \item \recup 3'00''
                    
                    \item \serie 3 $\times$ 4'00''
                    \\\vitesse Seuil$^+$
                    \\\pause 2'30'' (marche + trot)
                \end{itemize}
            }
        
        \subsubsection*{Résultats de la séance}
            Pour les 300 m à VMA, j'avais pour objectif 15.5 km/h, soit 1'09''.\\
            J'ai fait même mieux : 1'08'' pour les trois premiers tours... mais 1'10'' et 1'12'' pour les deux derniers.\\
            Je suis très proche des 16 km/h, mais c'est en même temps difficile à tenir sur la durée pour l'instant...\\\\
            %
            Pour les 4 minutes, je visais 1'19'' au tour soit 13 km/h.\\
            J'ai fait 1'17'' puis 1'19'' puis 1'20'' pour la première répétition.\\
            Puis 1'20'' à chacun des 3 tours pour la deuxième.\\
            Et pour la troisième, légèrement catastrophique, car j'ai fait 1'21'' deux fois puis 1'26'' (je n'ai donc pas pu finir le 3ème tour dans les 4 minutes imparties sur cette répétition).\\
            C'est sans aucun doute l'exercice le plus difficile pour moi.
        
        
    \subsection{10$^{\text{ème}}$ séance (04.04)}
        \subsubsection*{Description de la séance}
            \clovisSeance{
                \underline{\textbf{Objectifs :}}
                \begin{itemize}
                    \item Préparation aux 2 tests de l'UE Libre (à allure VMA et Seuil$^+$)
                    \item Passer de l'allure VMA à Seuil$^+$ pour se sentir plus léger
                    \item Se prouver qu'on peut courir (au total) 12 minutes à Seuil$^+$, même \textit{après} plusieurs 300 m à VMA
                \end{itemize}
                (Identique à la \underline{\hyperref[sec:seance9]{séance 9 {\scriptsize \faIcon{external-link-alt}}}}).\\\\
                %
                \underline{\textbf{Programme de la séance :}}
                \begin{itemize}
                    \item \serie 5 $\times$ 300 m
                    \\\vitesse VMA
                    \\\pause 1' (marche)
                    
                    \item \recup 3'00''
                    
                    \item \serie 3 $\times$ 4'00''
                    \\\vitesse Seuil$^+$
                    \\\pause 2'30'' (marche + trot)
                \end{itemize}
            }
        
        \subsubsection*{Résultats de la séance}
            Pour les 300 m à VMA, toujours le même objectif de 15.5 km/h, soit 1'09''.\\
            J'ai fait : 1'08'' pour les trois premiers tours puis deux fois 1'10''.\\
            Je suis toujours très proche des 16 km/h... si je suis en forme le jour du test (la semaine suivante), je pense essayer de courir à 16 km/h.\\\\
            %
            Pour les 4 minutes, je visais toujours 1'19'' au tour soit 13 km/h.\\
            J'ai fait 1'17'' puis 1'20'' puis 1'20'' pour la première répétition.\\
            Puis 1'20'' à chacun des 3 tours pour la deuxième.\\
            Et pour la troisième, j'ai aussi fait 1'20'' pour les 3.\\
            Je me stabilise bien à 1'20'', même si je redoute le test : ici, il y a une pause entre les 3 répétitions de 4 minutes...
        
        
    \subsection{11$^{\text{ème}}$ séance, test VMA n°2 (07.04)}
        %TODO
        \subsubsection*{Description de la séance}
            \clovisSeance{
                \underline{\textbf{Objectifs :}}
                \begin{itemize}
                    \item Test : tenir sa VMA pendant 4'00''
                    \item Courir à une VMA supérieure à celle du test précédent (viser 1 km de plus)
                \end{itemize}
                %
                \underline{\textbf{Programme de la séance :}}
                \begin{itemize}
                    \item \serie 1 $\times$ 4'00'
                    \\\vitesse VMA
                    
                    \item \recup 15'00'' 
                    
                    \item \serie 5 $\times$ 280 m 
                    \\\vitesse Seuil$^+$
                    \\\pause 1' (marche)
                \end{itemize}
            }
        
        \subsubsection*{Résultats de la séance}
            Allure prévue : 16 km/h, soit 45'' le tour\\\\
\begin{tabularx}{1\textwidth} { 
  | >{\raggedright\arraybackslash}X 
  | >{\raggedleft\arraybackslash}X 
  | >{\raggedleft\arraybackslash}X 
  | >{\raggedleft\arraybackslash}X 
  | >{\raggedleft\arraybackslash}X 
  | >{\raggedleft\arraybackslash}X | }
 \hline
 Distance{\scriptsize (m)} & 200 & 400 & 600 & 800 & 1000 \\
 \hline
 Temps & 0'41''  & 1'26'' & 2'14'' & 3'04'' & 3'49''  \\
 \hline
 Laps & 41'' & 45'' & 48'' & 50'' & 45''  \\
\hline
\end{tabularx}\\\\
        Distance parcourue : 1 075 m. (arrondi au plot supérieur)\\
        1.075 $\times$ 15 = 16.125, je suis donc à 16.125 km/h de VMA.\\\\
        Observation : objectif accompli, même légèrement plus mais je suis parti trop vite.\\
        Sur les 50 m de l'échauffement, j'étais bien à la bonne allure... mais je pense que je me suis senti \textit{trop} à l'aise avec cette vitesse, au point d'aller trop vite au début, en plus du stress du test.\\\\
        Le deuxième et le dernier tours étaient impeccables, j'ai eu une grosse perte de régime à 15 puis 14.5 km/h aux tours 3 et 4, mais j'ai pu reprendre de la vitesse au dernier tour.\\
        Ce test indique clairement que je suis capable de tenir une VMA de 16 km/h.\\
        J'ai donc progressé par rapport au test précédent.
        
        
        
    \subsection{12$^{\text{ème}}$ séance, test Seuil$^{\text{+}}$ (14.04)}
        %TODO
        \subsubsection*{Description de la séance}
            \clovisSeance{
                \underline{\textbf{Objectifs :}}
                \begin{itemize}
                    \item Test : tenir l'allure Seuil$^+$ pendant 12'00''
                    \item Courir à une vitesse plus rapide qu'aux premières séances
                \end{itemize}
                %
                \underline{\textbf{Programme de la séance :}}
                \begin{itemize}
                    \item \serie 1 $\times$ 12'00'' 
                    \\\vitesse Seuil$^+$
                \end{itemize}
            }
        
        \subsubsection*{Résultats de la séance}
            Allure prévue : 1'17'' au tour, soit 13.5 km/h (pour coller plus à ma nouvelle VMA de 16, même si je faisais 13 avant).\\\\
\begin{tabularx}{1\textwidth} { 
  | >{\raggedright\arraybackslash}X 
  | >{\raggedleft\arraybackslash}X 
  | >{\raggedleft\arraybackslash}X 
  | >{\raggedleft\arraybackslash}X 
  | >{\raggedleft\arraybackslash}X 
  | >{\raggedleft\arraybackslash}X | }
 \hline
 Nb tours & 1 & 2 & 3 & 4 & 5 \\
 \hline
 Temps & 1'10''  & 2'28'' & 3'50'' & 5'11'' & 6'30''  \\
 \hline
 Laps & 1'10'' & 1'18'' & 1'22'' & 1'21'' & 1'19''  \\
\hline
\end{tabularx}\\\\
\begin{tabularx}{1\textwidth} { 
  | >{\raggedright\arraybackslash}X 
  | >{\raggedleft\arraybackslash}X 
  | >{\raggedleft\arraybackslash}X 
  | >{\raggedleft\arraybackslash}X 
  | >{\raggedleft\arraybackslash}X | }
 \hline
 Nb tours & 6 & (2 km) 7 & 8 & 9 \\
 \hline
 Temps & 7'48''  & 9'08'' & 10'27'' & 11'40''  \\
 \hline
 Laps & 1'18'' & 1'20'' & 1'19'' & 1'13''  \\
\hline
\end{tabularx}\\\\
            J'ai fait environ 85 m dans les 20s qu'il me restait (plus d'un quart de tour), soit 2 660 m au total.\\
            Cela correspond à une vitesse de 13.3 km/h.\\
            Je suis très proche de mon objectif, très légèrement en-dessous.\\\\
            J'ai démarré trop vite (14.5 km/h soit 1 km/h trop vite), puis j'ai fait le bon temps au deuxième tour.\\
            (Comme pour le test des 4 minutes, j'attribue ce démarrage au stress du test : pour les 50 m à l'échauffement, j'avais bien ma bonne allure de 13.5 km/h.)\\\\
            Cependant, j'ai un peu perdu en vitesse aux tours 3 et 4 (je n'avais plus trop de souffle, sans doute car je suis parti trop vite au début).\\
            Puis je me suis stabilisé 4 tours d'affilée à 1'19''... ce qui correspond à 13 km/h, un peu en-dessous de ma vitesse prévue.\\
            J'ai compensé ce léger retard en accélérant à 14 km/h sur le dernier tour.\\\\
            Je suis satisfait de ma performance, puisque j'ai progressé : j'étais en-dessous de la barre des 13 km/h au début de l'UE Libre.
\section{Ressenti personnel et progression}
    J'ai effectué au total 11 séances.\\\\
    %
    J'ai beaucoup appris sur la façon de courir.\\
    J'ai enfin compris quel était le bon mouvement à adopter, courir sur la pointe des pieds, sans poser le talon, en utilisant le mouvement du talons-fesses à partir de VMA/VMA$^+$.\\
    J'ai commencé durant les séances d'UE Libre Running à intégrer petit à petit ces techniques dans ma course, et j'ai désormais un meilleur mouvement qu'au mois de janvier. Je compte continuer à améliorer ma technique jusqu'à ne plus poser le talon du tout, et utiliser au mieux mes jambes pour maximiser ma vélocité.\\\\
    J'ai aussi définitivement compris les repères sur la piste, à quoi correspondent les différentes distances, 50 m, 200 m, 300 m, etc.\\\\
    %
    Avant, je courais sans comprendre les séances de Running Loisir, maintenant je saisis mieux l'objectif de chacune des séances.\\
    Je saurais désormais travailler dans le cadre d'un objectif personnel, j'ai construit mes propres repères (différentes allures, avec le volume, l'intensité et la récupération qui vont avec).\\
    %
    J'ai appris à me repérer tout seul sur les différents tableaux de vitesse/temps, afin de me fixer des objectifs et j'ai aussi intégré les différentes allures et je sais à quoi elles correspondent pour moi.\\\\
    %
    Je pense être globalement meilleur sur du court que du sur long, il faudrait donc à l'avenir qu'à la fois je comble mes faiblesses en m'entraînant à un pourcentage de ma VMA sur un temps de course plus élevé, et que je table sur mes affinités avec le court, en faisant par exemple du fractionné très court à VMA$^{++}$.\\\\
    Voici le tableau qui résume mes performances :\\\\
\begin{tabularx}{1\textwidth} { 
  | >{\raggedright\arraybackslash}X 
  | >{\raggedleft\arraybackslash}X 
  | >{\raggedleft\arraybackslash}X 
  | >{\raggedleft\arraybackslash}X 
  | >{\raggedleft\arraybackslash}X | }
 \hline
 Allure & Footing & Seuil$^+$ & VMA & VMA$^+$\\
 \hline
 Vitesse{\scriptsize (km/h)} & 11 & 13.5 & 16 & 16, viser 17 \\
\hline
\end{tabularx}\\\\
    %
    J'ai réussi à atteindre 16 de VMA au test des 4 minutes, cela veut dire que je peux désormais viser une VMA$^+$ de 17 km/h.\\
    Comme je me suis plutôt senti à l'aise sur les 200 m, je pense que cela est atteignable prochainement.\\
    Il faut aussi que je consolide mes allures lentes afin de viser 11.5 et 14 pour les allures Footing et Seuil$^+$ respectivement.
\end{document}
