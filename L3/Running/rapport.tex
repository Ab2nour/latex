\documentclass{article}%
\usepackage[T1]{fontenc}%
\usepackage[utf8]{inputenc}%
\usepackage{lmodern}%
\usepackage{textcomp}%
\usepackage{lastpage}%
%
\usepackage{soul} % highlighting
\usepackage{inconsolata} % monospace font
\usepackage{fontawesome5}
%
\usepackage{xcolor}
\usepackage[framemethod=tikz]{mdframed}
\usepackage{tikzpagenodes}
\usetikzlibrary{calc}
%
% -------------------- Colors --------------------
\definecolor{definition-color}{HTML}{2f80ed}
\definecolor{definition-bg-color}{HTML}{e0ecfd}
\definecolor{danger-color}{HTML}{e6505f}
\definecolor{danger-bg-color}{HTML}{fce5e7}
\definecolor{code-bg-color}{HTML}{fcfcfc} % todo: temp color
\definecolor{code-border-color}{HTML}{dadce0} % todo: temp color
\definecolor{hl-yellow-color}{HTML}{fef3c7}
% -------------------- Macros --------------------
% highlight function
\newcommand{\highlight}[2]{%
    \begingroup
    \sethlcolor{#1}%
    \hl{#2}%
    \endgroup
}
\newcommand{\hlYellow}[1]{%
    \highlight{hl-yellow-color}{#1}%
}
% inlineCode (without border)
\newcommand{\inlineCodeWithoutBorder}[1]{%
    {\small\tt \highlight{code-bg-color}{#1}}%
}
% inlineCode (with border)
\usepackage[most]{tcolorbox}
\tcbset{
    on line,
    boxsep=2px,
    left=0pt,
    right=0pt,
    top=0pt,
    bottom=0pt,
    boxrule=0.5px,
    colframe=code-border-color,
    colback=code-bg-color,
    highlight math style={enhanced},
    breakable
}
\newcommand{\inlineCode}[1]{%
    \tcbox{{\small\tt #1}}%
}
% -------------------- colorful-blocks --------------------
\mdfdefinestyle{definition-style}{%
  innertopmargin=10px,
  innerbottommargin=10px,
  linecolor=definition,
  backgroundcolor=definition-bg,
  roundcorner=4px
}

\newcommand\clovisColorfulBlock[2]{
    % #1 = danger (name)
    % #2 = Danger (color)
    % #3 = danger-bg-color (background color)
    \mdfdefinestyle{#1-style}{%
        innertopmargin=12px,
        innerbottommargin=12px,
        innerleftmargin=12px,
        innerrightmargin=12px,
        linecolor=#1-color,
        backgroundcolor=#1-bg-color,
        roundcorner=6px
    }
    \newmdenv[style=#1-style]{#1}
    \expandafter\newcommand\csname clovis#2\endcsname[1]{
        \begin{#1}
        {\scriptsize \textcolor{#1-color}{\faIcon{running} \textbf{SÉANCE}}}
        \vspace{6px}
        \\ ##1
        \end{#1}
    }
}
\clovisColorfulBlock{definition}{Seance}



\newcommand{\titre}[1]{%
    \MakeUppercase{ \texttt{#1}}%
}

\newcommand{\serie}{%
    \titre{Série   : }
}

\newcommand{\vitesse}{%
    \titre{Vitesse : }%
}

\newcommand{\pause}{%
    \titre{Pause   : }%
}
% -------------------- Study Sheet --------------------
\title{Carnet d'Entraînement\\
UE Libre Running}%
\author{MADANI Abdenour}%
\date{\today}%
\normalsize%
%
\setcounter{tocdepth}{4}
\setcounter{secnumdepth}{4}
%
\begin{document}%
\normalsize%
\maketitle%
\tableofcontents%
\newpage%

\maketitle

\section{Introduction}
    Ceci est le rapport d'UE libre running.
    
    
    définition de la VMA

\section{Échauffement}
    20 minutes de footing
    gammes (montée de genoux, talons fesses, vite relâché vite)
    expliquer que c'est pour avoir le bon mouvement quand on court
    (rebondir sur la pointe des pieds et pas le talon)
    5 lignes droites

\section{Séances}
    \subsection{20.01}
    objectif trouver sa vitesse
    
    \clovisSeance{
        \underline{\textbf{Objectif :}}
        \begin{itemize}
            \item Chercher sa VMA
        \end{itemize}
        %
        \underline{\textbf{Programme de la séance :}}
        \begin{itemize}
            \item \serie 6 $\times$ 36'' 
            \\\vitesse VMA$^+$
            \\\pause 36'' (marche)
            
            \item Série : 5 $\times$ 1'12'' 
            \\Vitesse : VMA
            \\Pause : 1' (marche + trot)
            
            \item Série : 2 $\times$ 4' 
            \\Vitesse : Seuil$^+$
            \\Pause : 3' (marche)
        \end{itemize}
    }
    
    \subsection{27.01}
        absent (malade)
    
    \subsection{03.02}
    
    
    \subsection{10.02}
        test 4 minutes 1 km
    
    
    \subsection{17.02}
        300/250 m à VMA+ (16 KM)
        1min02
        Dernier tour 56 s
    
    
    \subsection{03.03}
        Running 3*5 * 200 m
        47 s
        Max 45 
        Min 49 (hum)
        
        À VMA plus (16 km h)
        
        
    \subsection{10.03}
        13 tours et demi (objectif de 13 au début)
        à 11 km/h (recalculer, vitesse seuil ) pendant 20 minutes
    
    
    \subsection{17.03}
        Ici objectif tenter plus que VMA (0.5 ou 1 km/h)
        300 m vitesse 15/16
        1.09 min 1.07 max 1.11
        250 m vitesse idem
        55 min 57 en moyenne max 59
        
        
    \subsection{24.03}
        5 * 300 VMA
        1.08 sauf deux 1.10 et 1.12
        
        Puis 3 * 4 minutes seuil+ (13)
        Objectif 1.19 voire 1.17 au tour
        2 fois objectif accompli, une fois boffff
        
        
    \subsection{04.04}
        Idem que 24.03
        
        
    \subsection{07.04}
        test 4 minutes
        observations : objectif accompli, mais manque de régularité
        je suis parti trop vite
        
\section{Ressenti personnel et progression}
    Donner mes impressions, mon ressenti
    
    
    dire où j'en étais *avant* et ce que j'ai tiré de l'UE libre :
    - j'ai beaucoup appris sur la façon de courir
    les différentes allures
    le bon mouvement (j'ai un meilleur mouvement désormais)
    j'ai définitivement compris les repères sur la piste
    
    avant je courais sans comprendre les séances de Running Loisir, maintenant je saisis mieux l'objectif de chaque séance
    Je saurais désormais travailler dans le cadre d'un objectif personnel
\end{document}