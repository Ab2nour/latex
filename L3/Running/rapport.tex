\documentclass{article}


% -------------------- Packages --------------------
\usepackage[utf8]{inputenc}
\usepackage[T1]{fontenc}%
\usepackage{lmodern}%
\usepackage{textcomp}%
\usepackage{lastpage}%


\usepackage{amsmath}
\usepackage{amssymb}
\usepackage{graphicx}
\usepackage{float}


\usepackage{soul} % highlighting
\usepackage{inconsolata} % monospace font
\usepackage{minted}
\usepackage{fontawesome5}


\usepackage{xcolor}
\usepackage[framemethod=tikz]{mdframed}
\usepackage{tikzpagenodes}
\usetikzlibrary{calc}


\usepackage{hyperref}
\hypersetup{
    colorlinks=true,
    linkcolor=blue,
    filecolor=magenta,      
    urlcolor=cyan,
    pdftitle={Clovis LaTeX Package},
    pdfpagemode=FullScreen,
    }

\urlstyle{same}

\usepackage{bookmark}
\hypersetup{hidelinks}

\usepackage{tabularx}

% -------------------- Colors --------------------
\definecolor{seance-color}{HTML}{2f80ed}
\definecolor{seance-bg-color}{HTML}{dfebfc}

\definecolor{example-color}{HTML}{6b7280}
\definecolor{example-bg-color}{HTML}{eceef2}

\definecolor{byheart-color}{HTML}{ffb0da}
\definecolor{byheart-bg-color}{HTML}{fff3f9}

\definecolor{danger-color}{HTML}{e6505f}
\definecolor{danger-bg-color}{HTML}{fbe4e7}

\definecolor{definition-color}{HTML}{47c2bb}
\definecolor{definition-bg-color}{HTML}{e3f5f4}

\definecolor{code-bg-color}{HTML}{f3f4f6} % todo: temp color
\definecolor{code-border-color}{HTML}{dadce0} % todo: temp color

\definecolor{hl-yellow-color}{HTML}{fef3c7}


% -------------------- Macros --------------------
% highlight function
\newcommand{\highlight}[2]{%
    \begingroup
    \sethlcolor{#1}%
    \hl{#2}%
    \endgroup
}
\newcommand{\hlYellow}[1]{%
    \highlight{hl-yellow-color}{#1}%
}
% inlineCode (without border)
\newcommand{\inlineCodeWithoutBorder}[1]{%
    {\small\tt \highlight{code-bg-color}{#1}}%
}
% inlineCode (with border)
\usepackage[most]{tcolorbox}
\tcbset{
    on line,
    boxsep=2px,
    left=0pt,
    right=0pt,
    top=0pt,
    bottom=0pt,
    boxrule=0.5px,
    colframe=code-border-color,
    colback=code-bg-color,
    highlight math style={enhanced},
    breakable
}
\newcommand{\inlineCode}[1]{%
    \tcbox{{\small\tt #1}}%
}
% -------------------- colorful-blocks --------------------
\mdfdefinestyle{definition-style}{%
innertopmargin=10px,
innerbottommargin=10px,
linecolor=definition-color,
backgroundcolor=definition-bg-color,
linewidth=2px,
topline=false,
bottomline=false,
rightline=false,
}

\newcommand\clovisColorfulBlock[4]{
    % #1 = danger (name)
    % #2 = Danger (color)
    % #3 = danger-bg-color (background color)
    \mdfdefinestyle{#1-style}{%
        innertopmargin=10px,
        innerbottommargin=10px,
        innerrightmargin=12px,
        linecolor=#1-color,
        backgroundcolor=#1-bg-color,
        linewidth=2px,
        topline=false,
        bottomline=false,
        rightline=false,
    }
    \newmdenv[style=#1-style]{#1}
    \expandafter\newcommand\csname clovis#2\endcsname[1]{
        \begin{#1}
        {\scriptsize \textcolor{#1-color}{\faIcon{#3} \textbf{#4}}}
        \vspace{6px}
        \\ ##1
        \end{#1}
    }
}
\clovisColorfulBlock{seance}{Seance}{running}{SÉANCE}
\clovisColorfulBlock{definition}{Definition}{graduation-cap}{DÉFINITION}



\newcommand{\titre}[1]{%
    \MakeUppercase{ \texttt{#1}}%
}

\newcommand{\serie}{%
    \titre{Série   : }
}

\newcommand{\vitesse}{%
    \titre{Vitesse : }%
}

\newcommand{\pause}{%
    \titre{Pause   : }%
}

\newcommand{\recup}{%
    \titre{Récup.  : }%
}
% -------------------- Study Sheet --------------------
\title{Carnet d'Entraînement\\
UE Libre Running}%
\author{MADANI Abdenour}%
\date{\today}%
\normalsize%
%
\setcounter{tocdepth}{4}
\setcounter{secnumdepth}{4}
%
\begin{document}%
\normalsize%
\maketitle%
\tableofcontents%
\newpage%

\maketitle

\section{Introduction}
    Ceci est mon carnet d'entraînement pour l'UE Libre Running.\\
    Je vais commencer par décrire un échauffement typique.\\
    Je parlerai ensuite dans le détail de chacune des séances, dont je décrirai le programme, puis les résultats que j'ai eus lors de cette séance, ainsi que mon ressenti.\\
    Je ferai ensuite un bilan de tout ce que j'ai appris au long de cette UE.\\\\
    Je définis ici un terme central pour cette UE, et la course en général, et qui reviendra très souvent par la suite, la \textbf{VMA}.
    \clovisDefinition{
                {\underline{\textbf{VMA}}\vspace{3px}}\\
                Vitesse Maximum Aérobie.\\
                C'est la plus petite vitesse à partir de laquelle la consommation d'oxygène est maximale.
    
    }

\section{Échauffement}
    On fait 12-20 minutes à allure footing : je commence toujours très lentement, afin de m'échauffer et de monter en allure progressivement.\\\\
    Je fais généralement 2 tours à 2'00'' (9 km/h), avant de monter petit à petit jusqu'à 1'35'' (11 km/h) au tour, rythme que j'arrive à tenir 20 minutes, comme je le décris dans la \underline{\hyperref[sec:seance7]{séance 7 {\scriptsize \faIcon{external-link-alt}}}}.\\\\
    %
    On continue par des gammes : montée de genoux, talons-fesses, "vite relâché vite" (3 à 5 fois).\\
    La montée de genoux et le talons-fesse servent à gagner en vélocité, mais en pratique, sur nos allures, on n'utilisera que le mouvement du talons-fesse, à partir de VMA$^+$.\\
    Le "vite relâché vite" sert à nous habituer à changer d'allure et à mieux nous rendre compte des variations de vitesse, notamment afin de mieux reprendre de la vitesse quand on commence à perdre du régime en fin de série.\\\\
    %
    Optionnellement, on fait parfois des lignes droites en "rebonds" (pour bien apprendre à rebondir sur la pointe des pieds), ou aussi des 50 m à l'allure désirée pour le début de la séance, afin de bien retrouver nos repères.\\\\
    %
    La récupération en fin de séance consiste toujours en une dizaine de minutes à allure footing.%
%

\section{Séances}
    \subsection{1$^{\text{ère}}$ séance (20.01)}
        \subsubsection*{Description de la séance}
            \clovisSeance{
                \underline{\textbf{Objectif :}}
                \begin{itemize}
                    \item Chercher sa VMA
                \end{itemize}
                %
                \underline{\textbf{Programme de la séance :}}
                \begin{itemize}
                    \item \serie 6 $\times$ 36'' 
                    \\\vitesse VMA$^+$
                    \\\pause 36'' (marche)
                    
                    \item \serie 5 $\times$ 1'12'' 
                    \\\vitesse VMA
                    \\\pause 1' (marche + trot)
                    
                    \item \serie 2 $\times$ 4'00''
                    \\\vitesse Seuil$^+$
                    \\\pause 3' (marche)
                \end{itemize}
            }
        
        \subsubsection*{Résultats de la séance}
            Ayant déjà fait du running en loisir pendant 1 semestre et demi, j'ai une petite idée de mes repères. Toutefois, je n'ai jusqu'à présent pas forcément compris tout ce que je faisais, et je suivais plutôt les autres.\\\\
            Je pense que ma VMA se situe autour de 15 km/h.\\\\
            %
            Je tente avec la VMA$^+$ à 16 km/h sur 36'' (= 0.01 h), objectif 160 m (donc 1 demi-tour + 20 m).\\
            J'ai un très bon ressenti, c'est sur cet exercice que je me sens le plus à l'aise.\\
            J'arrive exactement au même endroit à chaque fois, le temps de récupération est serré mais suffisant. Je m'en sors bien.\\\\
            %
            Pour les séries à VMA (15 km/h) : je suis arrivé au plot 15 (300 m) les trois premières fois, puis entre les plots 14 et 15 les deux dernières fois.\\
            C'est correct au début, mais je commence à lâcher prise passé la troisième fois...\\\\
            %
            Pour le Seuil$^+$, c'est encore pire : je suis théoriquement à 13.5 km/h (1'16'' au tour), et je devrais donc faire légèrement plus de 3 tours... cependant, j'ai fait moins de 3 tours, avec 1'25'' au tour, soit 12 km/h.\\
            Le ressenti était atroce, j'étais au bord de l'échec, pour un résultat insatisfaisant. Certes, c'est la fin de séance, mais je sens clairement que plus la durée de course est élevée, plus je suis en difficulté.
    %
    \subsection{2$^{\text{ème}}$ séance (27.01)}
        Absent (malade).
    
    \subsection{3$^{\text{ème}}$ séance (03.02)}
        \subsubsection*{Description de la séance}
            \clovisSeance{
                \underline{\textbf{Objectif :}}
                \begin{itemize}
                    \item Confirmer sa VMA
                \end{itemize}
                %
                \underline{\textbf{Programme de la séance :}}
                \begin{itemize}
                    \item \serie 5 $\times$ 1'12'' 
                    \\\vitesse VMA
                    \\\pause 1' (marche)
                    
                    \item \recup 3'00'' 
                    
                    \item \serie 5 $\times$ 1'12'' 
                    \\\vitesse VMA
                    \\\pause 1' (marche)
                \end{itemize}
            }
        
        \subsubsection*{Résultats de la séance}
    
    
    \subsection{4$^{\text{ème}}$ séance, test VMA n°1 (10.02)}
        \subsubsection*{Description de la séance}
            \clovisSeance{
                \underline{\textbf{Objectif :}}
                \begin{itemize}
                    \item Test : tenir sa VMA pendant 4 minutes
                \end{itemize}
                %
                \underline{\textbf{Programme de la séance :}}
                \begin{itemize}
                    \item \serie 1 $\times$ 4'00'' 
                    \\\vitesse VMA
                    
                    \item \recup 15'00'' 
                    
                    \item \serie 5 $\times$ 200 m
                    \\\vitesse VMA$^+$
                    \\\pause temps de course (marche)
                \end{itemize}
            }
        
        \subsubsection*{Résultats de la séance}
        test 4 minutes 1 km
        sensations : la gorge !
    
    
    \subsection{5$^{\text{ème}}$ séance (17.02)}
    \label{sec:seance5}
        %TODO
        \subsubsection*{Description de la séance}
            \clovisSeance{
                \underline{\textbf{Objectif :}}
                \begin{itemize}
                    \item Commencer sur du 300 m, puis continuer sur du 250 m à la même vitesse, pour se sentir plus à l'aise/pousser plus loin à VMA$^+$.
                    \item Augmenter sa VMA en tentant une allure plus rapide que sa VMA sur 300 m (la distance de course habituelle pour la VMA)
                \end{itemize}
                %
                \underline{\textbf{Programme de la séance :}}
                \begin{itemize}
                    \item \serie 4 $\times$ 300 m
                    \\\vitesse VMA$^+$
                    \\\pause temps - 10''
                    
                    \item \recup 2'30'' 
                    
                    \item \serie 4 $\times$ 250 m
                    \\\vitesse VMA$^+$
                    \\\pause temps - 10''
                    
                    \item \recup 2'30'' 
                    
                    \item \serie 4 $\times$ 250 m
                    \\\vitesse VMA$^+$
                    \\\pause temps - 10''
                \end{itemize}
            }
        
        \subsubsection*{Résultats de la séance}
            300/250 m à VMA+ (16 KM)
            1min02
            Dernier tour 56 s
    
    
    \subsection{6$^{\text{ème}}$ séance (03.03)}
        %TODO
        \subsubsection*{Description de la séance}
            \clovisSeance{
                \underline{\textbf{Objectif :}}
                \begin{itemize}
                    \item TODO
                \end{itemize}
                %
                \underline{\textbf{Programme de la séance :}}
                \begin{itemize}
                    \item \serie 5 $\times$ 200 m
                    \\\vitesse VMA$^+$
                    \\\pause temps de course (marche)
                    
                    \item \recup 3'00'' 
                    
                    \textbf{Le tout $\times$3}
                \end{itemize}
            }
        
        \subsubsection*{Résultats de la séance}
        Running 3*5 * 200 m
        47 s
        Max 45 
        Min 49 (hum)
        
        À VMA plus (16 km h)
        
        
    \subsection{7$^{\text{ème}}$ séance (10.03)}
    \label{sec:seance7}
        %TODO
        \subsubsection*{Description de la séance}
            \clovisSeance{
                \underline{\textbf{Objectif :}}
                \begin{itemize}
                    \item TODO
                \end{itemize}
                %
                \underline{\textbf{Programme de la séance :}}
                \begin{itemize}
                    \item \serie 1 $\times$ 20'00''
                    \\\vitesse Footing
                    
                    \item \recup 5'00'' 
                    
                    \item \serie 5 $\times$ 1'12'' 
                    \\\vitesse VMA
                    \\\pause 1' (marche)
                \end{itemize}
            }
        
        \subsubsection*{Résultats de la séance}
        13 tours et demi (objectif de 13 au début)
        à 11 km/h (recalculer, vitesse seuil ) pendant 20 minutes
        le mental compte beaucoup (pas que le physique)
        couru à plusieurs (effet de boost !)
    
    \subsection{8$^{\text{ème}}$ séance (17.03)}
        %TODO
        \subsubsection*{Description de la séance}
            \clovisSeance{
                \underline{\textbf{Objectif :}}
                \begin{itemize}
                    \item Commencer sur du 300 m, puis continuer sur du 250 m à la même vitesse, pour se sentir plus à l'aise/pousser plus loin à VMA$^+$.
                    \item Augmenter sa VMA en tentant une allure plus rapide que sa VMA sur 300 m (la distance de course habituelle pour la VMA)
                \end{itemize}
                (Identique à la \underline{\hyperref[sec:seance5]{séance 5 {\scriptsize \faIcon{external-link-alt}}}}).\\\\
                %
                \underline{\textbf{Programme de la séance :}}
                \begin{itemize}
                    \item \serie 4 $\times$ 300 m
                    \\\vitesse VMA$^+$
                    \\\pause temps - 10''
                    
                    \item \recup 2'30'' 
                    
                    \item \serie 4 $\times$ 250 m
                    \\\vitesse VMA$^+$
                    \\\pause temps - 10''
                    
                    \item \recup 2'30'' 
                    
                    \item \serie 4 $\times$ 250 m
                    \\\vitesse VMA$^+$
                    \\\pause temps - 10''
                \end{itemize}
            }
        
        \subsubsection*{Résultats de la séance}
        Ici objectif tenter plus que VMA (0.5 ou 1 km/h)
        300 m vitesse 15/16
        1.09 min 1.07 max 1.11
        250 m vitesse idem
        55 min 57 en moyenne max 59
        
        
    \subsection{9$^{\text{ème}}$ séance (24.03)}
    \label{sec:seance9}
        \subsubsection*{Description de la séance}
            \clovisSeance{
                \underline{\textbf{Objectif :}}
                \begin{itemize}
                    \item Préparation aux 2 tests de l'UE Libre (à allure VMA et Seuil$^+$)
                    \item Passer de l'allure VMA à Seuil$^+$ pour se sentir plus léger
                    \item Se prouver qu'on peut courir (au total) 12 minutes à Seuil$^+$, même \textit{après} plusieurs 300 m à VMA
                \end{itemize}
                %
                \underline{\textbf{Programme de la séance :}}
                \begin{itemize}
                    \item \serie 5 $\times$ 300 m
                    \\\vitesse VMA
                    \\\pause 1' (marche)
                    
                    \item \recup 3'00''
                    
                    \item \serie 3 $\times$ 4'00''
                    \\\vitesse Seuil$^+$
                    \\\pause 2'30'' (marche + trot)
                \end{itemize}
            }
        
        \subsubsection*{Résultats de la séance}
        5 * 300 VMA
        1.08 sauf deux 1.10 et 1.12
        
        Puis 3 * 4 minutes seuil+ (13)
        Objectif 1.19 voire 1.17 au tour
        2 fois objectif accompli, une fois boffff
        
        
    \subsection{10$^{\text{ème}}$ séance (04.04)}
        \subsubsection*{Description de la séance}
            \clovisSeance{
                \underline{\textbf{Objectif :}}
                \begin{itemize}
                    \item Préparation aux 2 tests de l'UE Libre (à allure VMA et Seuil$^+$)
                    \item Passer de l'allure VMA à Seuil$^+$ pour se sentir plus léger
                    \item Se prouver qu'on peut courir (au total) 12 minutes à Seuil$^+$, même \textit{après} plusieurs 300 m à VMA
                \end{itemize}
                (Identique à la \underline{\hyperref[sec:seance9]{séance 9 {\scriptsize \faIcon{external-link-alt}}}}).\\\\
                %
                \underline{\textbf{Programme de la séance :}}
                \begin{itemize}
                    \item \serie 5 $\times$ 300 m
                    \\\vitesse VMA
                    \\\pause 1' (marche)
                    
                    \item \recup 3'00''
                    
                    \item \serie 3 $\times$ 4'00''
                    \\\vitesse Seuil$^+$
                    \\\pause 2'30'' (marche + trot)
                \end{itemize}
            }
        
        \subsubsection*{Résultats de la séance}
        Idem que 24.03
        
        
    \subsection{11$^{\text{ème}}$ séance, test VMA n°2 (07.04)}
        %TODO
        \subsubsection*{Description de la séance}
            \clovisSeance{
                \underline{\textbf{Objectif :}}
                \begin{itemize}
                    \item Test : tenir sa VMA pendant 4'00''
                    \item Courir à une VMA supérieure à celle du test précédent (viser 1 km de plus)
                \end{itemize}
                %
                \underline{\textbf{Programme de la séance :}}
                \begin{itemize}
                    \item \serie 1 $\times$ 4'00'
                    \\\vitesse VMA
                    
                    \item \recup 15'00'' 
                    
                    \item \serie 5 $\times$ 280 m 
                    \\\vitesse Seuil$^+$
                    \\\pause 1' (marche)
                \end{itemize}
            }
        
        \subsubsection*{Résultats de la séance}
        test 4 minutes
        observations : objectif accompli, mais manque de régularité
        je suis parti trop vite
        
        
    \subsection{12$^{\text{ème}}$ séance, test Seuil$^{\text{+}}$ (14.04)}
        %TODO
        \subsubsection*{Description de la séance}
            \clovisSeance{
                \underline{\textbf{Objectif :}}
                \begin{itemize}
                    \item Test : tenir l'allure Seuil$^+$ pendant 12'00''
                    \item Courir à une vitesse plus rapide qu'aux premières séances
                \end{itemize}
                %
                \underline{\textbf{Programme de la séance :}}
                \begin{itemize}
                    \item \serie 5 $\times$ 1'12'' 
                    \\\vitesse VMA
                    \\\pause 1' (marche)
                    
                    \item \recup 3'00'' 
                \end{itemize}
            }
        
        \subsubsection*{Résultats de la séance}
        test 12 minutes
        
\section{Ressenti personnel et progression}
    J'ai effectué au total 11 séances.
    Donner mes impressions, mon ressenti
    
    
    dire où j'en étais *avant* et ce que j'ai tiré de l'UE libre :
    - j'ai beaucoup appris sur la façon de courir
    les différentes allures
    le bon mouvement (j'ai un meilleur mouvement désormais)
    j'ai définitivement compris les repères sur la piste
    
    avant je courais sans comprendre les séances de Running Loisir, maintenant je saisis mieux l'objectif de chaque séance
    Je saurais désormais travailler dans le cadre d'un objectif personnel, j'ai construit mes propres repères (différentes allures, avec le volume, l'intensité et la récupération qui vont avec)
    
    j'ai appris à me repérer tout seul sur les différents tableaux
    
    
    dire les sensations (manque de souffle ? jambes courbaturées ?)
    et aussi dire que je me sens meilleur au sprint / sur du court
    que sur du long, donc que plus tard
    je devrais travailler mes faiblesses sur le long/endurance
    et renforcer mes facilités en sprint tout ça
    
    le plus dur pour moi est de maintenir un pourcentage de ma VMA longtemps => je dois travailler ça
    
    
    mettre un tableau avec ma vitesse pour chaque allure différente vue (et les repères associées, genre la distance) ?
\end{document}
