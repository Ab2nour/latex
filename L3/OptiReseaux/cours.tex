\documentclass{article}
\usepackage[utf8]{inputenc}

\title{Optimisation dans les réseaux}
\author{aut}
\date{2022}

\begin{document}

\maketitle

\section{Introduction}
% p 5/51 livre blanc
$a_i$ = taille de l'objet i
\\%
$y_i$ :
\\- 0 $\rightarrow$ la boîte n'est pas utilisée
\\- 1 $\rightarrow$ la boîte est utilisée
\\%
$x_{i, j}$ :
\\- 0 $\rightarrow$ l'objet i n'est pas dans la boîte j
\\- 1 $\rightarrow$ l'objet i est dans la boîte j
\\%
$x_{1, 1} + x_{1, 2} + ... + x_{1, 7} = 1$ $\rightarrow$ cela veut dire que l'objet 1 est dans \textbf{une} seule boîte
\\%
On cherche à minimiser le nombre de boîtes, donc :
$$\min \sum_{j=1}^{7} y_j$$

Adjacence : entre deux éléments de même nature
\\Incidence : entre deux éléments de nature différente
\end{document}
