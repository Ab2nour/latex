\documentclass{article}
\usepackage[T1]{fontenc}
\usepackage[utf8]{inputenc}
\usepackage[dvipsnames]{xcolor}
\usepackage{lmodern}
\usepackage{textcomp}
\usepackage{lastpage}
\usepackage{tikz}
\usepackage{amsmath}
\usepackage{graphicx}
\usepackage{float}
\setcounter{secnumdepth}{0}

\definecolor{exogris}{gray}{0.4}

\title{Mathématiques Financières}
\author{Licence 3}
\date{2021 - 2022}

\begin{document}

\normalsize
\maketitle

\renewcommand*\contentsname{Table des matières}
%\tableofcontents
\newpage

\section{1   Capitalisation dans le moyen terme}
\subsection{Exercice n°1}
\textcolor{exogris}{
Un épargnant décide de déposer la somme $C_1 = 1200$ \texteuro sur un compte rémunéré au taux d’intérêt annuel $i_1 = 0.015$, le 27 janvier 2020.
}
\\\\ \textcolor{exogris}{\textbf{1.1.}
Quelle sera la valeur acquise au 27 janvier 2022 ?
}
\\%
Ma réponse
\\%
\\%
\textcolor{exogris}{\textbf{1.2.}
Il pense pouvoir ajouter le 27 janvier 2021 un complément $C_2 =$ 500 \texteuro. Combien obtiendra-t-il le 27 janvier 2022 ?
}
\\%
Ma réponse
\\%
\\%
\textcolor{exogris}{\textbf{1.3.}
En se plaçant dans la situation précédente, quel devrait être le taux d’intérêt $i_2$ pour obtenir une valeur
acquise de 1773.25 \texteuro le 27 janvier 2022 ?
}%
Ma réponse


\subsection{Exercice n°2}
\textcolor{exogris}{
Le capital $C$ = 900 \texteuro est immobilisé pendant 5 ans.
}
\\\\ \textcolor{exogris}{\textbf{2.1.}
Avec un taux annuel d’intérêt $i$ = 1.8\%, quel sera le montant disponible $C_5$ à l’échéance ?
}
\\%
Ma réponse
\\%
\\%
\textcolor{exogris}{\textbf{2.2.}
En fait, le taux d’intérêt annuel est égal à $i_1$ = 1\% les deux premières années et à $i_2$ = 2\% les trois dernières. Préciser quel sera le nouveau montant $C^{'}_5$ à la fin des 5 années.
}
\\%
Ma réponse
\\%
\\%
\textcolor{exogris}{\textbf{2.3.}
Quel taux annuel moyen $i_{moy}$ pourrait-on associer à la deuxième situation ?
}%
\\%
Ma réponse


\end{document}
