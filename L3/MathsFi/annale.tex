\documentclass{article}
\usepackage[T1]{fontenc}
\usepackage[utf8]{inputenc}
\usepackage[dvipsnames]{xcolor}
\usepackage{lmodern}
\usepackage{textcomp}
\usepackage{lastpage}
\usepackage{tikz}
\usepackage{amsmath}
\usepackage{graphicx}
\usepackage{float}
%\setcounter{secnumdepth}{0}

\definecolor{exogris}{gray}{0.4}

\title{Mathématiques Financières}
\author{Licence 3}
\date{2021 - 2022}

\begin{document}

\normalsize
\maketitle

\renewcommand*\contentsname{Table des matières}
%\tableofcontents
\newpage

\section{Capitalisation dans le moyen terme}
\textcolor{exogris}{\textbf{Exercice n°1}
\\Un épargnant décide de déposer la somme $C_1 = 1200$ \texteuro sur un compte rémunéré au taux
d’intérêt annuel i1 = 0.015, le 27 janvier 2020.}
\end{document}
