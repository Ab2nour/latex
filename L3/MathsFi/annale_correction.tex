\documentclass{article}
\usepackage[T1]{fontenc}
\usepackage[utf8]{inputenc}
\usepackage[dvipsnames]{xcolor}
\usepackage{lmodern}
\usepackage{textcomp}
\usepackage{lastpage}
\usepackage{tikz}
\usepackage{amsmath}
\usepackage{graphicx}
\usepackage{float}
\setcounter{secnumdepth}{0}

\definecolor{exogris}{gray}{0.4}

\title{Mathématiques Financières}
\author{Licence 3}
\date{2021 - 2022}

\begin{document}

\normalsize
\maketitle

\renewcommand*\contentsname{Table des matières}
%\tableofcontents
\newpage

\section{1   Capitalisation dans le moyen terme}
% ---------- Exercice 1 ----------
\subsection{Exercice n°1}
\textcolor{exogris}{
Un épargnant décide de déposer la somme $C_1 = 1200$ \texteuro sur un compte rémunéré au taux d’intérêt annuel $i_1 = 0.015$, le 27 janvier 2020.
}
\\\\ \textcolor{exogris}{\textbf{1.1.}
Quelle sera la valeur acquise au 27 janvier 2022 ?
}
\\%
D'après le cours, la valeur acquise au bout de $n$ années $V_n$, avec un capital initial $C$ et un taux d'intérêt $i$ est :
$$V_n = C\cdot(i+1)^n$$
Donc la valeur acquise $V_2$ au 27 janvier 2022, 2 ans après est :
$$V_2 = 1200 \cdot 1.015^2 = \boxed{1236.27 \text{ \texteuro}}$$
\\%
\\%
\textcolor{exogris}{\textbf{1.2.}
Il pense pouvoir ajouter le 27 janvier 2021 un complément $C_2 =$ 500 \texteuro. Combien obtiendra-t-il le 27 janvier 2022 ?
}
\\%
On décompose les deux années en deux périodes :
\\\\\textbf{Période 1} : janvier 2020 $\rightarrow$ janvier 2021
\\Le capital de départ est toujours $C_1$. On ré-applique la formule sur 1 an donc avec $n=1$, avec $V_1$ le capital obtenu en janvier 2021 soit :
$$V_1 = C_1 \cdot (i+1)$$
\\\textbf{Période 2} : janvier 2021 $\rightarrow$ janvier 2022
\\Le nouveau capital de départ est $V_1 + C_2$.
Le taux reste inchangé, on ré-applique la formule pour trouver $V_2$ acquis 1 an plus tard:
$$V_2 = (V_1 + C_2) \cdot (i+1)^n $$

On remplace $V_1$ par sa valeur $C_1 \cdot (i+1)$
$$ V_2 = (C_1 \cdot (i+1) + C_2)\cdot (i+1)$$
$$\implies V_2 = C_1 \cdot (i+1)^{2} + C_2 \cdot (i+1)$$
$$= 1200 \cdot 1.015^2 + 500 \cdot 1.015 = \boxed{1743.77 \text{ \texteuro}}$$
L'épargnant obtiendra donc 1743.77 \texteuro.
\\%
\\%
\textcolor{exogris}{\textbf{1.3.}
En se plaçant dans la situation précédente, quel devrait être le taux d’intérêt $i_2$ pour obtenir une valeur
acquise de 1773.25 \texteuro le 27 janvier 2022 ?
}%
\\%
On cherche à obtenir $i$ en connaissance des autres termes d'après la formule générale.
D'après la question précédente on a :
$$\implies V_2 = C_1 \cdot (i+1)^{2} + C_2 \cdot (i+1)$$
$$\implies C_1 \cdot (i+1)^2 + C_2 \cdot (i+1) - V_2 = 0$$

% ---------- Exercice 2 ----------
\subsection{Exercice n°2}
\textcolor{exogris}{
Le capital $C$ = 900 \texteuro est immobilisé pendant 5 ans.
}
\\\\ \textcolor{exogris}{\textbf{2.1.}
Avec un taux annuel d’intérêt $i$ = 1.8\%, quel sera le montant disponible $C_5$ à l’échéance ?
}
\\%
On raisonne de la même façon que pour l'exercice \textbf{1.1}.
$$C_5 = C\cdot(i+1)^5 \implies C_5 = 900 \cdot 1.018^5 = \boxed{983.97\text{ \texteuro}}$$
\\%
\\%
\textcolor{exogris}{\textbf{2.2.}
En fait, le taux d’intérêt annuel est égal à $i_1$ = 1\% les deux premières années et à $i_2$ = 2\% les trois dernières. Préciser quel sera le nouveau montant $C^{'}_5$ à la fin des 5 années.
}
\\%
On raisonne de la même façon que pour l'exercice \textbf{1.2}.
On décompose les 5 ans en deux périodes :
\\\\\textbf{Période 1} : Les 2 premières années (année 0 $\rightarrow$ année 2)
\\Le capital de départ est toujours $C$. On ré-applique la formule sur 2 ans donc avec $n=2$, avec $C_2$ le capital obtenu en début d'année 2 :
$$C_2 = C \cdot (i_1+1)^2$$
\\\textbf{Période 2} : Les 3 dernières années (année 2 $\rightarrow$ année 5)
\\Le nouveau capital de départ est $C_2$. On prend $n = 3$.
$$C^{'}_5 = C_2\cdot(i_2+1)^3$$
$$= C\cdot(i_1+1)^2\cdot(i_2+1)^3$$
$$\implies C^{'}_5 = 900\cdot1.01^2\cdot1.02^3 = \boxed{974.28\text{ \texteuro}}$$
%
Le nouveau montant $C'_5$ sera donc de 974.28 \texteuro à la fin des 5 années.
\\%
\\%
\textcolor{exogris}{\textbf{2.3.}
Quel taux annuel \textit{moyen} $i_{moy}$ pourrait-on associer à la deuxième situation ?
}%
\\%
On cherche à isoler $i$ dans la formule : 
$$V_n = C\cdot(i+1)^n \implies i = \left(\dfrac{V_n}{C}\right)^{\frac{1}{n}}-1$$
d'où :
$$i_{moy} = \left(\dfrac{C^{'}_5}{C}\right)^{\frac{1}{n}}-1 = \left(\dfrac{974.28}{900}\right)^{\frac{1}{5}}-1 = \boxed{0.0168}$$
On pourrait associer à la deuxième situation un taux annuel moyen de 1.68\%.


% ---------- Exercice 3 ----------
\subsection{Exercice n°3}
\textcolor{exogris}{
Un investisseur se propose de verser annuellement auprès d’un organisme financier 12000 \texteuro  du 01/01/2020 au 01/01/2023.
}
\\\\ \textcolor{exogris}{\textbf{3.1.}
Préciser la valeur acquise, $V_{acq}$, au 01/01/2024 si le placement est rémunéré au taux annuel $i$ = 0.75\%.
}
\\%
D'après le cours, en ayant des versements constants de valeur $a$ en début de période, avec un taux d'intérêt $i$ sur $n$ périodes (avec ici $a = 12000$ et $n = 4$), on a :
$$V^{deb}_{acq} = a(i+1)\cdot\dfrac{(i+1)^n-1}{i}$$
$$\implies V_{acq} = 12000\cdot  1.0075\cdot\dfrac{1.0075^4-1}{0.0075} = \boxed{48906.78\text{ \texteuro}}$$
\\%
\\%
\textcolor{exogris}{\textbf{3.2.}
Réalisant la difficulté de réunir en une seule fois la somme de 12000 \texteuro , il envisage de déposer mensuellement la somme de 1000 \texteuro  à partir du 01/01/2020 et jusqu’au 01/12/2023.
}
\\%
\textcolor{exogris}{\textbf{3.2.1}
Quel est le taux mensuel $i_m$ équivalent au taux $i$ ? Le résultat sera donné avec 6 chiffres après la virgule.
}%
\\%
Le taux mensuel $i_m$ se doit de vérifier, avec $C$ capital de départ quelconque :
$$C(i_m + 1)^{12} = C(i+1)$$
d'où
$$\boxed{i_m = (i+1)^{\frac{1}{12}} - 1} = 1.0075^{\frac{1}{12}}-1 = \boxed{0.000623}$$
Le taux mensuel $i_m$ équivalent est de $0.000623$.
\\%
\\%
\textcolor{exogris}{\textbf{3.2.2}
Déterminer la valeur acquise $V^{'}_{acq}$ de ces versements mensuels à la date 01/01/2024 lorsque les intérêts sont capitalisés au taux mensuel $i_m$. Que remarquez-vous ?
}%
\\%
Du 01/01/2020 jusqu'au 01/12/2023 il y a 48 mois.
$$V^{deb}_{acq} = a(i+1)\cdot\dfrac{(i+1)^n-1}{i}$$
\\Application numérique : on a donc ici $n$ = 48, $i$ = $i_m$ = 0.000623 et $a$ = 1000.
Soit :
$$V'_{acq} = 1000 \cdot 1.000623\cdot\dfrac{1.000623^{48}-1}{0.000623} = \boxed{48739.85\text{ \texteuro}}$$
\\On remarque que les deux placements ne sont pas équivalents : la stratégie annuelle est plus fructueuse que la stratégie mensuelle ($V_{acq} > V'_{acq}$).




\section{2   Actualisation dans le moyen terme}
% ---------- Exercice 4 ----------
\subsection{Exercice n°4}
\textcolor{exogris}{
Un capital de $C$ = 5000 \texteuro  est disponible le 27 janvier 2023. Le taux d’actualisation $\tau$ est fixé à 6.5\% pour toutes les questions.
}
\\\\ \textcolor{exogris}{\textbf{4.1.}
Déterminer la valeur actuelle, au 27 janvier 2020, de ce capital.
}
\\%
D'après le cours, on a, avec $V_{act}$ la valeur actualisée, $V_f$ le capital futur disponible, $n$ le nombre d'années et $\tau$ le taux d'actualisation :
$$V_{act} \times(\tau+1)^n= V_f \implies \boxed{V_{act} = \dfrac{V_f}{(\tau+1)^n}}$$
Ici on a $V_f = C = 5000 \text{ \texteuro}$.
%
\\On procède à l'application numérique :
$$V_{act} = \dfrac{5000}{(1.065)^3} = \boxed{4139.25\text{ \texteuro}}$$
La valeur actuelle de ce capital, au 27 janvier 2020, est donc 4139.25 \texteuro.
\\%
\\%
\textcolor{exogris}{\textbf{4.2.}
Quand serait-il disponible si sa valeur actuelle à la date du 27 janvier 2020 était égale à 3 217.53 \texteuro ?
}
\\%
On cherche ici à exprimer $n$, le nombre d'années, en fonction des autres paramètres. Reprenons la formule : 
$$V_{act} = \dfrac{V_f}{(\tau+1)^n} \implies (\tau+1)^n = \dfrac{V_f}{ V_{act}} \implies n \ln(\tau+1) = \ln\left(\dfrac{V_f}{ V_{act}}\right)$$
$$\implies \boxed{n = \dfrac{\ln(V_f) - \ln(V_{act})}{\ln(\tau+1)}}$$
On effectue l'application numérique :
$$n = \dfrac{\ln(5000) - \ln(3217.53)}{\ln(1.065)} = \boxed{7.00 \text{ ans}}$$
Le capital serait dans ce cas disponible 7 ans plus tard.

% ---------- Exercice 5 ----------
\subsection{Exercice n°5}
\textcolor{exogris}{
Une entreprise étudie les deux investissements ci-dessous :
\begin{center}
\begin{tabular}{ |c|c|c|c|c| } 
 \hline
 Années & 0 & 1 & 2 & 3 \\ \hline
 Projet 1 & -70000 & 30000 & 45000 & 40000 \\ \hline
 Projet 2 & -60000 & 15000 & 40000 & 50000 \\
 \hline
\end{tabular}
\end{center}
}%
 
%
\\ \textcolor{exogris}{\textbf{5.1.}
Proposer deux formules calculant respectivement les valeurs actualisées nettes du projet 1, $VAN_1(\tau)$, et du projet 2, $VAN_2(\tau)$, en fonction du taux d’actualisation $\tau$.
}
\\%
D'après le cours on a :
$$VAN(\tau) = -V_0 + \sum_{i=1}^n \dfrac{V_i}{(1+\tau)^{p_i}}$$
d'où :
$$\boxed{VAN_1(\tau) = -70000 + \dfrac{30000}{(1+\tau)} + \dfrac{45000}{(1+\tau)^2} + \dfrac{40000}{(1+\tau)^3}}$$
et 
$$\boxed{VAN_2(\tau) = -60000 + \dfrac{15000}{(1+\tau)} + \dfrac{40000}{(1+\tau)^2} + \dfrac{50000}{(1+\tau)^3}}$$
\\%
\\%
\textcolor{exogris}{\textbf{5.2.}
Calculer ces valeurs pour $\tau$ = 10\%. Quel projet conseillez-vous ?
}
\\%
En prenant $\tau = 0.1$, on a :
$$VAN_1(0.1) = -70000 + \dfrac{30000}{1.1} + \dfrac{45000}{(1.1)^2} + \dfrac{40000}{(1.1)^3} = \boxed{24515.40 \text{ \texteuro}}$$
et 
$$VAN_2(0.1) = -60000 + \dfrac{15000}{(1.1)} + \dfrac{40000}{(1.1)^2} + \dfrac{50000}{(1.1)^3} = \boxed{24259.95 \text{ \texteuro}}$$
On a $VAN_1(0.1) > VAN_2(0.1)$, donc le projet 1 est légèrement plus rentable sur un horizon de 3 ans.
\\%
\\%
\textcolor{exogris}{\textbf{5.3.}
Préciser $VAN_1(0.28)$. Qu’en pensez-vous ?
}%
\\%
En prenant $\tau = 0.28$, on a :
$$VAN_1(0.28) = -70000 + \dfrac{30000}{1.28} + \dfrac{45000}{(1.28)^2} + \dfrac{40000}{(1.28)^3} = \boxed{-23.19 \text{ \texteuro}}$$
Le projet n'est donc plus rentable avec un taux de 28\%.





\end{document}
